\section*{Text Viewers}

As we remember, UNIX was originally created to automate the work of
the patent office, has a rich set of tools for working with text data.
But what is text? Generally it is just a collection of bytes encoded
according to some encoding table, originally ASCII. In a text file,
you will not see any special formatting like bold text, italics,
images, etc.~--- just text data. And this is the main communication format
for UNIX utilities since the 1970s.

As you know, Microsoft operating systems have different modes of working
with files~--- text and binary. In UNIX, all files are the same, and
we have no difference between text and binary data. See details in
%[``Under the Hood'']
(\url{under_the_hood/pipelining_Windows.txt}).

\subsection*{Concatenating and spliting}

The first creature that helps us work with text files is the ``\cmd{cat}''.
Not a real ``cat'', but an abbreviation for concatenation. With no arguments,
cat simply copies standard input to standard output. And as we understand it,
we can just redirect the output to a file, and this will be the easiest way
to create a new file:
\begin{code}{mverb}
cat > file
\end{code}
When we add filenames as arguments to our command, this will be a real
concatenation~--- they will all be sent to the output. And if we redirect them
to a file, we get all these files concatenated into an output file.
\begin{code}{mverb}
cat f1 f2... > all
\end{code}

If we can combine something, we must be able to split it.
And we have two utilities for different types of breakdowns:
\begin{itemize}
\item \cmd{tee}~--- read from standard input and write to standard output and
      files
\item \cmd{split}~--- split a file into fixed-size pieces
\end{itemize}

\subsection*{Text viewers and editors}

What is it viewer? In the TTY interface, the man command seems like
a good one~--- when you run it, you get paper manuals that you can combine
into a book, put on a shelf, and reread as needed. On a full-screen terminal~---
before, Ctrl-S (stop)/Ctrl-Q(repeat) was enough for viewing, because at first
the terminals were connected at low speed (9600 bits per second for ex.),
and now special programs were used~--- viewers. Unlike text editors,
viewers does not need to read the entire file before starting, resulting in
faster load times with large files.

Historically, the first viewer was the ``\cmd{more}'' pager developed for
the BSD project in 1978 by Daniel Halbert, a graduate student at the University
of California, Berkeley. The command-syntax is:
\begin{code}{mverb}
more [options] [file_name]
\end{code}
If no file name is provided, `more' looks for input from standard input.

Once `more' has obtained input (file or stdin), it displays as much as can fit
on the current screen and waits for user input. The most common methods of
navigating through a file are Enter, which advances the output by one line,
and Space, which advances the output by one screen.  When `more' reaches
the end of a file (100\%) it exits. You can exit from ``more'' by pressing
the ``q'' key and the ``h'' key will display help. In the `more' utility
you can search with regular expressions using the `slash' or the `+/' option.
And you can search again by typing just a slash without regexp.
Regexp is a very important part of UNIX culture and is used in many other
programs and programming environments:
%[Pr-n 6 slide 4]
(\url{http://sdn.ifmo.ru/education/courses/free-libre-and-open-source-software/lectures/lecture-6})

The `main' limitation of the more utility is only forward movement in the text.
To solve this problem, an improved `more' called `\cmd{less}' was developed.
The ``less'' utility buffers standard input, and we can navigate forward and
backward through the buffer, for example. using the cursor keys or
the PgUp/PgDown keys. A reverse search with a question mark is possible.
