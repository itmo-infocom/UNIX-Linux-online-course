\section*{Your system}

The best advantage of free systems is their \struct{availability}. You can
download many kinds of Linux or UNIX systems for free. You can distribute
such systems, downloaded from the Internet, and use them to create your
own customized solutions using a huge number of already existing
components.

For example, for educational purposes we use our own Linux distribution
called \struct{NauLinux}. This title is the abbreviated title of the Russian
translation of the original \struct{Scientific Linux} project~---
``Nauchnyi Linux''. We are adding many software packages for working
software-defined networks and quantum cryptography emulating and are using
this new distributions to simulate various complex systems in educational or
research projects. This distribution is free and is used by students to
create their own solutions.

\struct{Scientific Linux} on which we are based is also free. It was created at
Fermilab for use in high energy physics and has focused from the beginning
on creating specialized flavours optimized for the needs of laboratories
and universities.

This, in turn, is based on \struct{Red Hat Enterprise Linux} (RHEL),
a commercial Linux distribution widely used in the industry. You will be
charged for using the binaries for that distribution and getting support
from the vendor. But the sources from this distro are still free, and anyone
can download it and rebuild their own distro.

The source of RHEL, in turn, is the \struct{Fedora experimental project},
developed by the community with the support of Red Hat. Leveraging large
communities of skilled and motivated users lowers the cost of testing,
development, and support for the company. And this is an example of
the profitable use of the Free and Open Source model by a commercial company.

There are many free BSD OS variants, currently \struct{FreeBSD}, \struct{NetBSD},
\struct{OpenBSD}, \struct{DragonFly BSD}. They all have their own specifics and
their own kernels with incompatible system calls. This is a consequence of
the fact that the development of these systems is driven by communities in
which disagreements arise from time to time, and they are divided according
to different interests regarding the development of systems.

On the Linux project, we still have one and only one benevolent dictator,
\struct{Linus Torvalds}. As a result, we still have a single main kernel
development thread published on \url{kernel.org}. While many other experimental
Linux kernel flavours are also being developed, not all of them are
accepted into the mainstream. In turn, on the basis of this single
kernel, the development of various OS distributions is made, often with
some changes from the distribution vendors.

The most commonly used free Linux distributions are:
\begin{itemize}
\item community driven \struct{Debian} project
\item \struct{Ubuntu} Ditro based on Debian and developing by Canonical company
\item The \struct{Fedora} Project on which RHEL development is based
\item and \struct{Centos}~--- another free RHEL respin
\end{itemize}
\struct{Gentoo}, \struct{Arch}, \struct{Alpine} and many other Linux
distributions are also well known. Many projects are focused on embedded systems,
such as BuildRoot, Bitbake, OpenWRT, OpenEmbedded and others.

You can usually use them in different ways - install on your computer
(including coexistence with other systems such as Windows, with the ability
to dual boot), boot from a live image or network server without installing
the OS to a local hard drive, run as a container or virtual machine on your
local computer or network cloud, etc. You can find detailed information
on installing and configuring such systems in the documentation of
the respective projects.

Moreover, you can simply use online services to access UNIX or Linux
systems, for example:
\begin{itemize}
\item \href{https://skn.noip.me/pdp11/pdp11.html}%
      {\url{https://skn.noip.me/pdp11/pdp11.html}}~--- PDP-11 emulator with UNIX
\item \href{https://bellard.org/jslinux}{\url{https://bellard.org/jslinux}}~---
      a lot of online Linux'es
\end{itemize}

When you log in, you will be asked for a user and password. Depending on
your system configuration, after logging in, you will have access to
a graphical interface or text console. In both cases, you will have access
to the \struct{Shell command line} interface, which we are most interested in.

The user password is set by the `root' superuser during installation or
system configuration, and this password can be changed by the user
himself or by the superuser for any user using the `\cmd{passwd}' command.
