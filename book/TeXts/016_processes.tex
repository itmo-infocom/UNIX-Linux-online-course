\section*{Processes}
%by Oleg Sadov — last modified Aug 21, 2020 05:00 PM

\subsection*{Process}

We've discussed the users, and then it's appropriate to talk about
another of the three whales of UNIX-like systems~--- processes. We can get
information about the processes by running the ``\cmd{ps}'' (process status)
command. In this case, we again see two worlds~--- two systems SYSV and BSD:
\begin{code}{mverb}
SYSV:  ps [-efl]
BSD:   ps [-][alx]
\end{code}

What about GNU? As we can see, GNU ps supports both sets of options with
some long options.

By default \cmd{ps} without options shows only process started by me and
connected to my current terminal line.

To get the status for all processes, we must use:
\begin{code}{mverb}
SYSV:  ps -ef
BSD:   ps ax
\end{code}

And we can get more information about the processes using the ``long'' options:
\begin{code}{mverb}
SYSV:  ps -l
BSD:   ps l
\end{code}

What information about the processes can we see?
\begin{description}
\item[\cmd{UID}]~--- effective user ID. A process can have a different
identifier than the user running the application because, as we will see
later, there is a mechanism in UNIX-like systems to change the identifier
on the run.
\item[\cmd{PID}]~--- a number representing the unique identifier of the process.
\item[\cmd{PPID}]~--- parent process ID. As we remember, we have
a hierarchical system of processes, and each process has its own parent.
We can see this hierarchy for example by such options set:
\begin{code}{mverb}
ps axjf | less
\end{code}
or just by command:
\begin{code}{mverb}
pstree
\end{code}
\item[\cmd{PRI}]~--- priority of the process. Higher number means lower
priority. But, as we will discuss later, we cannot change the priority,
because this value is dynamically changed by the process scheduler. And
we can only send recommendations to the scheduler using the 'nice' (NI)
parameter:
\item[\cmd{NI}]~--- can be set with 'nice' and 'renice' commands
\item[\cmd{TTY}]~--- controlling tty (terminal).
\item[\cmd{CMD}]~--- and the command.
\end{description}

And also a very useful (especially if the system hangs) command `\cmd{top}',
which dynamically displays information about processes, sorted
accordingly by the use of system resources~--- memory and CPU time.

{\bf\cmd{nice}}~--- run a program with modified scheduling priority.
`\struct{Nice}' value is just an integer. The smallest number means
the highest priority. The nice's range can be different on different systems
and you should look at the ``\cmd{man nice}'' on your system. In the case of
Linux nice value~--- between -20 and 19. Only the superuser can increase
the priority, the normal user can just decrease the default, which can be seen
by invoking the ``\cmd{nice}'' command with no arguments.

For example:
\begin{code}{mverb}
nice -n 19 command args...
\end{code}
means execution of the command with the lowest priority. This can be useful
for reducing the activity of non-interactive applications, such as the backup
process, which can slow down the interactive response of the system.

{\bf\cmd{renice}}~--- alter priority of running processes by PID. In this
case, you may not use the `\cmd{-n}' option~--- just a `nice' number.
For example:
\begin{code}{mverb}
renice 19 PID...
\end{code}

\subsection*{Jobs}

At the Shell level, we can use the `jobs' mechanism.

The easiest way to start a new background job is to use the ampersand (\cmd{\&}):
\begin{code}{mverb}
gedit &
xeyes
\end{code}

Once the command is running, we can disconnect from the terminal line and
pause it by pressing '\cmd{\^{}Z}'. As we can see, the eyes do not move now.

We can see background and suspended jobs using the jobs command:
\begin{code}{mverb}
jobs
\end{code}

In the first position of the jobs list, we see the job number. We can use
this job number with a percent sign in front of it.

A suspended task, we can switch it to the background execution mode. By
default~--- current job:
\begin{code}{mverb}
bg [%jobN] - resume suspended job jobN in the background
\end{code}
and reattach the background job to the terminal line by bringing it to
the foreground:
\begin{code}{mverb}
fg [%jobN] - resume suspended job jobN in the foreground
\end{code}
After that we can interrupt the foreground job by pressing `\cmd{\^{}C}'.

\subsection*{Signals}

Another way to terminate a process is with the `\cmd{kill}' command:
\begin{code}{mverb}
kill %job
\end{code}
also you can kill the process by PID number:
\begin{code}{mverb}
kill [-s sigspec | -n signum | -sigspec] [pid | jobspec] ...
\end{code}

But in some cases `\cmd{kill}' does not work~--- for example, if the process is
frozen. We can fix this problem by calling another kill, just because
kill is actually sending a signal to the process, and we just have to
choose a different signal.
\begin{code}{mverb}
kill -l -- list of signals
\end{code}

\begin{itemize}
\item[15)] \cmd{SIGTERM}~-- generic signal used to cause program termination
(default kill)\\
\item[2)] \cmd{SIGINT}~--- ``program interrupt'' (INTR key~--- usually Ctrl-C)
\item[9)] \cmd{SIGKILL}~--- immediate program termination (cannot be blocked,
handled or ignored)
\item[1)] \cmd{SIGHUP}~--- terminal line is disconnected (often used for daemons
config rereading)
\item[3)] \cmd{SIGQUIT}~--- core dump process (QUIT key -- usually C-\textbackslash)
\end{itemize}

\subsection*{Offline execution}

When you execute a Unix job in the background ( using \cmd{\&}, \cmd{bg} command),
and logout from the session, your process will get killed.
We can avoid this using nohup command:

\cmd{nohup}~--- run a command immune to hangups,
with output to `{\tt nohup.out}'

Another very useful program is `\cmd{screen}'~--- it's screen manager with
VT100/ANSI terminal emulation which supports multi-screen session support
with offline execution. In fact, you can run some long running commands
on multiple screen sessions and after disconnecting from this terminal
line with your hands or after breaking connecting. After that, you can
reconnect to this screen and you will see that all processes are still running.

\subsection*{Later execution and scheduled commands}

Another possibility of offline executing commands is later execution and
scheduled commands.

\cmd{at}, \cmd{batch}, \cmd{atq}, \cmd{atrm}~--- queue, examine or delete jobs
for later execution

\cmd{crontab}~--- maintain crontab files for individual users
