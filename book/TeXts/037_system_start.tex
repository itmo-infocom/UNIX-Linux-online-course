\section*{Booting and services starting/stopping} % ## Booting and services starting/stopping

All right. But how does our system get started? In fact, when you turn on
the computer, a special piece of code is launched, built into the hardware~--
firmware. There are many such firmware: \struct{legacy PC BIOS}, \struct{UEFI},
\struct{Uboot}, \struct{OpenBoot}, \struct{Coreboot}, etc. The firmware reads
the main bootloader from disk: \struct{BIOS} from \struct{MBR},
\struct{UEFI} from \struct{EFI system partition}, and so on.

Then the secondary bootloader started. This loader can be more or less complex
and customizable. The most commonly used Linux bootloaders currently are
lightweight boot system \cmd{SYSLINUX} for \struct{FAT} file system and
\cmd{ISOLINUX} for \struct{ISO} images, which is mainly used to boot
installation or live images, and the general purpose \struct{Grub bootloader}.

Usually Grub is installed during system installation, including configuring
the boot of other operating systems installed on your computer. But in some
cases MS Windows can reinstall the bootloader without asking you, in which case
you may lose your boot settings. To restore it, you need to boot from
the installation media in repair mode and run the `\cmd{grub2-install}' program
for your system storage. For example:
\begin{code}{mverb}
grub2-install /dev/sda
\end{code}

After the kernel is loaded, a special process called `\cmd{init}` is started.
In original UNIX, as well as BSD `\cmd{init}', just run the script `\cmd{/etc/rc}',
which completely determines the further behavior of the system.

A different `\cmd{init}' machinery is implemented for SYSV systems. On such
systems, you can invoke the `\cmd{init}' program at any time by setting
the runlevel as a parameter. Runlevels defined in the `\cmd{init}' configuration
is located in the \href{https://manpages.debian.org/unstable/sysvinit-core/inittab.5.en.html}%
{`\cmd{/etc/inittab}' file}.
%['/etc/inittab' 
%file](https://manpages.debian.org/unstable/sysvinit-core/inittab.5.en.html).
Each line in the \cmd{inittab} file consists of four colon-delimited fields and
describes:
\begin{itemize}
\item what processes to start, monitor, and restart if they terminate
\item what actions to take when the system enters a new runlevel
\item the default runlevel
\end{itemize}

Inittab's fields:
\begin{code}{mverb}
id:runlevels:action:process
\end{code}

\struct{id} (identification code)~--- consists of a sequence of one to four
characters that identifies its function.
\begin{itemize}
\item runlevels~--- lists the run levels to which this entry applies.
\item action~--- specific codes in this field tell init how to treat
      the process. Possible values include: \struct{initdefault},
      \struct{sysinit}, \struct{boot}, \struct{bootwait}, \struct{wait},
      and \struct{respawn}.
\item process~--- defines the command or script to execute.
\end{itemize}

The line `\cmd{initdefault}' defines the default runlevel. Different systems
define different init level hierarchies, but some of them have the same
meaning:
\begin{itemize}
\item Runlevel 0 is \struct{halt}.
\item Runlevel 6 is \struct{reboot}.
\item Runlevel 1 is \struct{single-user}.
\item \struct{2--5} are most often some multiuser runlevels.
\end{itemize}

Most often, the executable scripts in `\cmd{inittab}' are just some `\cmd{rc}'
scripts that go through the appropriate \cmd{/etc/rc<runlevel>.d/} directories
and run the stop scripts first, starting with a big \cmd{K} (kill) with
a `\cmd{stop}' parameter, and then starting the scripts that start
with large \cmd{S} with a `\cmd{start}' parameter:
\begin{code}{mverb}
$ ls -l /etc/rc.d/rc5.d/
\end{code} % $

And, as we can see, this script is simply symbolic links to scripts from
`\cmd{/etc/init.d/}', moreover, the Kill scripts and the Start scripts can be
linked to the same file. If we look at them, we will see~--- there are just
scripts that do something according to the start or stop parameter:
\begin{code}{mverb}
$ less /etc/init.d/network
\end{code} % $
And if you want to implement our own service script, you just have to support
the `\cmd{start}' and `\cmd{stop}' parameters. To configure your own policy
for stopping and starting services at any level, you simply have to link
the scripts which you need from `\cmd{/etc/init.d/}' to the appropriate
runlevel directories. The order in which scripts are run is determined by
the numbers at the beginning of the filenames.

Some commands that can help you with this work:
\begin{itemize}
\item `\cmd{service}'~-- run a System V init script
\item `\cmd{setup}' and `\cmd{chkconfig}'~--- updates and queries runlevel
       information for system services
\end{itemize}

The most commonly used Linux distributions now use `\cmd{systemd}' instead of
such systems. The main advantage of this system is faster parallel launch of
services at system startup, as well as unification of services and work with
devices. These utilities and scripts are still present for compatibility,
but now the main tool is `\cmd{systemctl}':
\begin{code}{mverb}
man systemctl
\end{code}
We can list system services, start, stop and get their status, enable and
disable them to automatically start at boot time.

To find out the log messages about boot startup and system operation,
we can look at the system log files:
\begin{itemize}
\item \cmd{/var/log/messages}~--- RH-like Linuxes
\item \cmd{/var/log/syslog}~--- Debian, Ubuntu
\end{itemize}
In modern Linux based on `\cmd{systemd}' we have `\cmd{journalctl}'
to work with the `\cmd{systemd}' journal system.
