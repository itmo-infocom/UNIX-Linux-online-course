Now let's try to localize our program.
\begin{code}{mverb}
$ git checkout Example_8
\end{code} % $
What is it "localize"? The locale is the primary mechanism for supporting the native languages in UNIX-like systems, and we have a few strange abbreviations that describe it: I18N, L10N, and even M17N. What does this mean? This means that English does not like long words, but the terminology associated with supporting native languages is too many letters. Such as -- Multilingualization for application software consisting of 17 letters between "M" and "N" and abbreviated to M17N. M17N is performed in 2 stages:
\begin{itemize}
\item Internationalization (18 letters between ``I'' and ``N''~--- I18N):
      To make a software potentially handle multiple locales.
\item Localization (10 letters between ``L'' and ``N''~--- L10N):
      To make a software handle an specific locale.
\end{itemize}

OK. Let's take a look at our current locale settings:
\begin{code}{mverb}
$ locale
...
\end{code} % $
As we can see, there are lot of environment variables associated with this:
\begin{itemize}
\item \cmd{LANG} and \cmd{LC\_ALL}~--- General language setting.
      These settings can be separated to:
\item \cmd{LC\_CTYPE}~--- Controls the behavior of character handling functions.
\item \cmd{LC\_TIME}~--- Specifies date and time formats, including month names,
      days of the week, and common full and abbreviated representations.
\item \cmd{LC\_MONETARY}~--- Specifies monetary formats, including the currency
      symbol for the locale, thousands separator, sign position, the number of
      fractional digits, and so forth.
\item \cmd{LC\_NUMERIC}~--- Specifies the decimal delimiter (or radix character),
      the thousands separator, and the grouping.
\item \cmd{LC\_COLLATE}~--- Specifies a collation order and regular expression
      definition for the locale.
\item \cmd{LC\_MESSAGES}~--- Specifies the language in which the localized
      messages are written, and affirmative and negative responses of the locale
      (yes and no strings and expressions).
\item \cmd{LO\_LTYPE}~--- Specifies the layout engine that provides information
      about language rendering. Language rendering (or text rendering)
      depends on the shape and direction attributes of a script.
\end{itemize}

In the C programming language, the locale name C ``specifies the minimal
environment for C translation'' (C99 \S7.11.1.1; the principle has been
the same since at least the 1980s). As most operating systems are written in C,
especially the Unix-inspired ones where locales are set through the \cmd{LANG}
and \cmd{LC\_xxx} environment variables, C ends up being the name of
a ``safe'' locale everywhere

On POSIX platforms such as Unix, Linux and others, locale identifiers are
defined by \struct{ISO} standard and consists from:
`\verb|[language[_territory][.codeset][@modifier]]|'. For example, Australian
English dialect using the UTF-8 encoding is \struct{en\_AU.UTF-8}.

As we can see, our application already has l10n.
\begin{code}{mverb}
$ LANG=C calc_ui
$ LANG=en_US.UTF-8 calc_ui
\end{code}

As we can see, our application looks the same in base C and American
English locales. In Russian we see the Russian title and buttons:
\begin{code}{mverb}
$ LANG=ru_RU.UTF-8 calc_ui
\end{code} % $
In Continental Simplified and Traditional Taiwanese Chinese, we can see
different hieroglyphs sets on the title:
\begin{code}{mverb}
$ LANG=zh_CN.UTF-8 calc_ui
$ LANG=zh_TW.UTF-8 calc_ui
\end{code}
And for the Arabic language, the title is written from right to left:
\begin{code}{mverb}
$ LANG=ar_SY.UTF-8 calc_ui
\end{code} % $
Looks good, but this is just a basic localization of the `\cmd{zenity}' utility
used by `\cmd{gdialog}'. We need to translate our text strings in the program.
And for that we can use the standard `\cmd{gettext}' machinery that was first
implemented by Sun Microsystems in 1993. To work with this, we need
a `\cmd{gettext}' utilities set:
\begin{code}{mverb}
# yum install gettext
# apt install gettext
\end{code}
We have a utility `\cmd{gettext}' that translates the text messages passed to it
as arguments according to the settings of the locale's environment variables:
\begin{code}{mverb}
$ man gettext
\end{code} % $
As we can see, for this we just need to add a call to the `\cmd{gettext}'
utility in all places where we used a text string as a parameter
to `\cmd{gettext}', and insert the result of execution into command lines
by quoting the command with ``back apostrophes'':
\begin{code}{mverb}
$ git diff Example_7 Example_8
\end{code} % $
And now we can fetch the gettext strings from the source using
the \cmd{xgettext} utility:
\begin{code}{mverb}
$ make calc_ui.pot
xgettext -o calc_ui.pot -L Shell calc_ui
\end{code} % $

We got the portable object template file `\cmd{calc\_ui.pot}' that we will use
as a basis for translation:
\begin{code}{mverb}
$ cat calc_ui.pot
...
\end{code} % $
For example to Russian language:
\begin{code}{mverb}
$ msginit --locale=ru --input=calc_ui.pot
...
$ cat ru.po
\end{code}
