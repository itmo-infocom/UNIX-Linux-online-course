\section*{Utilities}

All commands typed on the command line or executed in a command file are
either commands built into the interpreter or external executable files.
The set of built-in commands is quite small, which is determined by the
basic concept of UNIX~--- the system should consist of small programs that
perform fairly simple well-defined functions that communicate with each
other via a standard interface.

A rich set of standard utilities is a good old tradition for UNIX-like
systems. The shell and the traditional set of UNIX utilities,
is a \struct{POSIX} standard.

As we discussed earlier, we have different branches of development of
UNIX-like systems with different types of utilities:
\begin{itemize}
\item \struct{BSD}-like dating back to the original UNIX implementations;
\item \struct{SYSV} based systems;
\item \struct{GNU} utilities.
\end{itemize}

Some command syntax was changed by the USL with the introduction of SYSV,
but on most commercial UNIX a set of older commands was still included
for compatibility with earlier BSD-based versions of UNIX from the same vendor.
\struct{GNU} utilities often combine both syntaxes and add their own
enhancements to traditional utilities. And now the GNU toolkit has become
the de facto standard.

Executable files on UNIX-like systems do not have any file name extension
requirements as they do on Windows. The utility executable can have any name,
but must have execute permission for the user who wants to run it.

A standard utility can have options, argument of options, and operands.
Command line arguments of programs are mainly parsed by the getopt()
function, which actually determines the form of the parameters when
the command is invoked. This is an example of utility's synopsys description:

\begin{code}{mverb}
utility_name[-a][-b][-c option_argument] \
                          [-d|-e][-f[option_argument]][operand...]
\end{code}

\noindent
\begin{enumerate}
\item The utility in the example is named \struct{\cmd{utility\_name}}.
It is followed by \struct{options}, \struct{option-arguments}, and \struct{operands}.
The arguments that consist of <hyphen-minus> characters (`\struct{-}') and
single letters or digits, such as `a', are known as ``\struct{options}''
(or, historically, ``\struct{flags}''). Certain options are followed by
an ``\struct{option-argument}'', as shown with [-c option\_argument].
The arguments following the last options and option-arguments are
named ``\struct{operands}''.

The GNU \cmd{getopt()} function supports so-called long parameters, which start
with two dashes and can use the full or abbreviated parameter name:
\begin{code}{mverb}
utility_name --help
\end{code}
\item \struct{Option-arguments} are shown separated from their options by
<blank> characters, except when the option-argument is enclosed in the
'\struct{[}' and '\struct{]}' notation to indicate that it is optional.

In GNU getopt's long options also may be used the 'equal' sign between
option and option-argument:
\begin{code}{mverb}
utility_name --option argument --option=argument
\end{code}
\item When a utility has only a few permissible options, they are
sometimes shown individually, as in the example. Utilities with many
flags generally show all of the individual flags (that do not take
option-arguments) grouped, as in:
\begin{code}{mverb}
utility_name [-abcDxyz][-p arg][operand]
\end{code}
Utilities with very complex arguments may be shown as follows:
\begin{code}{mverb}
utility_name [options][operands]
\end{code}
\item Arguments or option-arguments enclosed in the '[' and ']'
notation are optional and can be omitted. Conforming applications shall
not include the '[' and ']' symbols in data submitted to the utility.
\item Arguments separated by the '\struct{|}' ( <vertical-line>) bar notation
are mutually-exclusive.
\begin{code}{mverb}
utility_name [-a|b] [operand...]
\end{code}
Alternatively, mutually-exclusive options and operands may be listed with
multiple synopsis lines. For example:
\begin{code}{mverb}
utility_name [-a] [-b] [operand...]
utility_name [-a] [-c option_argument] [operand...]
\end{code}
\item Ellipses ( ``\struct{...}'' ) are used to denote that one or more
occurrences of an operand are allowed.
\end{enumerate}
