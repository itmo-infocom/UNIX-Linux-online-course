\section*{Introduction to development} % ## Introduction to development

And now we are ready to do some kind of development project. We will try
to implement the following traditional task for novice developers~---
a calculator. And finally, we will get a network client-server application
with a graphical user interface localized into Russian. We can find the code
for this project on github:

\href{https://github.com/itmo-infocom/calc\_examples}%
     {https://github.com/itmo-infocom/calc\_examples}
%[https://github.com/itmo-infocom/calc_examples](https://github.com/itmo-infocom/calc_examples)

Github supports hosting for software development and version control with \struct{Git}.

\struct{Version control} (also known as \struct{revision control},
\struct{source control}, or \struct{source code management}) is a class of
systems responsible for managing changes to computer programs, documents,
large web sites, or other collections of information. Version control is
a component of software configuration management.

\href{https://initialcommit.com/blog/Technical-Guide-VCS-Internals}%
     {VCS Release History Timeline}
%[VCS Release History Timeline](https://initialcommit.com/blog/Technical-Guide-VCS-Internals)

The \struct{first generation} VCS were intended to track changes for individual
files and checked-out files could only be edited locally by one user at a time.
They were built on the assumption that all users would log into the same shared
Unix host with their own accounts. The \struct{second generation} VCS introduced
networking which led to centralized repositories that contained
the `\struct{official}' versions of their projects. This was good progress,
since it allowed multiple users to checkout and work with the code at the same
time, but they would all be committing back to the same central repository.
Furthermore, network access was required to make commits. The \struct{third generation}
comprises the distributed VCS. In a distributed VCS, all copies of
the repository are created equal~--- generally there is no central copy of
the repository. This opens the path for commits, branches, and merges to be
created locally without network access and pushed to other repositories
as needed.

There are many VCS systems developed and used on UNIX-like systems: \struct{SCCS},
\struct{RCS}, \struct{CVS}, \struct{SVN}, \struct{Mercurial}, \struct{Bazzar},
etc., and \struct{Git} is now the most popular. \struct{Git} is a distributed
version-control system for tracking changes in source code during software
development. It was created by Linus Torvalds in 2005 for development of
the Linux kernel, with other kernel developers contributing to its initial
development.

\subsection*{Devlopment tools} % ## Devlopment tools

The basic tool for working with Git repositories is the `\cmd{git}' utility:
\begin{code}{mverb}
man git
\end{code}
First we have to `\cmd{clone}' our repository:
\begin{code}{mverb}
$ git clone https://github.com/itmo-infocom/calc_examples
$ ls calc_examples/
calc           calc_ui      calc_ui-ru.po  gdialog   README.md
calc.services  calc_ui.pot  calc.xinetd    Makefile
\end{code}
Ok~--- we now have our own local copy. Let's go inside:
\begin{code}{mverb}
$ cd calc_examples/
\end{code} % $
Now let's take a look at the tags that indicate different stages of
development~--- tags:
\begin{code}{mverb}
$ git tag
Example_1
Example_2
Example_3
Example_4
Example_5
Example_6
Example_7
Example_8
Example_9
\end{code} % $
Let's move on to the initial stage now:
\begin{code}{mverb}
$ git checkout Example_1
Note: checking out 'Example_1'.
...
$ ls
Makefile  README.md
\end{code}
As we can see, most of the files have disappeared and now we only have
a README and a Makefile. Let's take a look at the README:
\begin{code}{mverb}
$ cat README.md 
calc_examples
...
\end{code} % $
and at the Makefile:
\begin{code}{mverb}
$ cat Makefile 
clone:
...
\end{code} % $
A \cmd{Makefile} is a file containing a set of directives used by a `\cmd{make}'
build automation tool to generate a target/goal. Make is the oldest and most
widely used dependency-tracking tool for building software projects.
It is used to build large projects such as the Linux kernel with tens of
millions of source code lines. Usually, using make is pretty simple:
you just run make with no arguments:
\begin{code}{mverb}
man make
\end{code}
In this case, the utility tries to find a makefile that starts with a lowercase
letter and then with a capital letter. Usually the second Makefile with
an uppercase letter is used, the lowercase version is often used for some local
changes while testing and customizing the Makefile of an upstream project.

The makefile format is pretty simple~--- It's just a lot of rules
for building projects:
\begin{code}{mverb}
$ cat Makefile 
\end{code} % $
In the first position we see the `\cmd{target}', after the semicolon,
dependencies can be placed, which are just other targets. Lines following
the description of targets and dependencies must start with a TAB character,
and there are commands that must be executed to complete this target.
If the dependencies are files, make checks the modification times of
those files and recursively rebuilds those dependency targets.

The current Makefile just includes a few targets for working with
the \struct{Git} repository. The first will start by default. If we want to run
another target, we must pass it as a parameter when running
the `\cmd{make}' utility.
