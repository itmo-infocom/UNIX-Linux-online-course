\section*{Main Concepts}

At the top level, UNIX-like systems can be very convenient for common
users, and they may not even know they are using this type of OS.
For example, currently the most commonly used operating systems are
Linux-based Android systems and UNIX-based Apple systems, in which
the user only sees the user friendly graphical UI.

But beginners who are just starting to learn UNIX-like systems
for administration or development sometimes complain about their complexity.
Don't be afraid~--- actually such systems are based on fairly simple concepts.
There are only three things (three and a half to be exact) you
need to know to be comfortable with any UNIX-like system:
\begin{itemize}
\item[1)] \struct{Users}
\item[2)] \struct{Files}
\item[3)] \struct{Processes}
\item[3.5)] \struct{Terminal lines}
\end{itemize}

The \struct{users} is not very well known to modern users only because we now
have a lot of computer devices with personal access. UNIX was created at a time
when computers were an expensive rarity and a single computer was used
by many users. As a consequence, from the beginning, \struct{UNIX} had
\struct{strong security policies} and \struct{restrictions on permissions}
for users.

And now on UNIX-like systems, we have dozens of users and groups,
even if hidden by an autologin machinery. And most of them are so-called
\struct{pseudo-users}, which are needed to start system services. As we will see
later, they are required by architecture, since it is on the permissions
of users and groups that the system is built to control access to system
resources (processes and files).

If we are talking about ordinary users, they can log in with a \struct{username}
and \struct{password} and interact with the applications installed on the system.
Each user has full permissions only in their home directory and limited
access rights to files and directories outside of it. This can be viewed
as foolproof~--- common users cannot destroy anything on the system just
because they do not have such permissions. Moreover, they \struct{cannot view}
another user's home directory or protected system files and directories.
To perform system administration tasks, the system has a special
\struct{superuser} (generally called ``\struct{root}'') with extra-permissions.

At the system level, each user or group looks like an integer number:
a~user identifier (\struct{UID}) and a group identifier (\struct{GID}).

\struct{Files} are the next important thing for UNIX-like systems. Almost
all system resources look like files, including devices and even
processes on some systems. And the basic concepts have been the same
since the beginning of the UNIX era. We have a hierarchical file system
with a single root directory. All resources, including file systems
existing on devices or external network resources, are attached to this
file system in separate directories~--- this operation is called
``mount''. On the other hand, you can access a device (real or virtual)
as a stream of bytes and work with it like a regular file. All files and
directories are owned by users (real or pseudo) and groups, and read,
write, and execute access to them is controlled by permissions.

A \struct{process} is a program launched from an executable file. Each process
belongs to a user and a group. The relationship between the owners of
processes and resources determines the access rights according to
the resource permissions. All processes live in a hierarchical system based
on parent-child relationships. There is an initial process on the system
called ``\cmd{init}'' that is started at boot. All system services are started
from this initial process.

There are fundamentally two types of processes in Linux~--- foreground and
background:
\begin{itemize}
\item \struct{Foreground processes} (also referred to as interactive processes)~---
      these are initialized and controlled through a terminal session.
      In other words, there has to be a user connected to the system to start
      such processes; they haven’t started automatically as part of the system
      functions/services.
\item \struct{Background processes} (also referred to as non-interactive/automatic
      processes)~--- are processes not connected to a terminal; they don’t
      expect any user input. System services are always background processes.
\end{itemize}

And finally~--- interactive foreground processes must be attached to
the terminal session through the terminal line. At the time of the creation of
UNIX, a TTY (teletype) device (originally developed in the 19th century),
was the primary communication channel between the user and the computer.
It was a very simple interface that worked with a stream of bytes encoded
according to the ASCII character set. The connection is made via a serial
interface (for example RS232) with a fixed set of connection speeds.

%This interface is still the main user interface for UNIX-like systems.
This interface is still the main user interface for UNIX-like systems.
The implementation of each new form of user interaction, such as
full-screen terminals, graphics systems, and network connections, all
started with the implementation of a simple TTY command line interface.
Moreover, as we will see, this interface abstraction gives us a very
powerful and flexible mechanism for communication between programs,
possibly without human intervention.
