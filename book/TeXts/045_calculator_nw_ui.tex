Ok, we have a working network service, and it's time to change our user
interface to network communication with it.
\begin{code}{mverb}
$ git checkout Example_7
$ git diff Example_6 Example_7 | less
\end{code}
The main changes are made in the `\cmd{calc\_ui}' script. At the beginning of
the script, as you can see, added some default definitions: \cmd{HOST},
\cmd{PORT} for calc-server connection and \cmd{CALC} script name.

Then we can see added support for configuration files. First, we check
the user's personal configuration in the home directory~--- this is a hidden
file starting with a ``dot'' \cmd{calc.conf}. If such a file exists, we load it
as a Shell library file. If we do not find this file, we check the system-wide
configuration `\cmd{/etc/calc.conf}' and load it if it exists.
You can put your own connection variable definitions in these files.

We see a new `\cmd{help}' function for displaying a help message and some logic
for processing script parameters. We now handle the `\cmd{-\mbox{}-help}'
parameter and the optional host/port positional parameters. As we can see,
we can only set the host or both the host and port using the script arguments.

And finally~--- we check the name of the script by stripping it with
the `\cmd{basename}' function from the current path to the script. If the script
is called `\cmd{ncalc\_ui}', we change the \cmd{CALC} variable to `\cmd{netcat}'
with the host access parameters. If such parameters are not specified,
we display an error message. In the line where we called the `\cmd{calc}' script,
we replace it with the \cmd{CALC} variable.

That's it~--- we just changed the UI file without changing the main logic.
Let's install it:
\begin{code}{mverb}
$ sudo make install
[sudo] password for liveuser: 
install calc calc_ui /usr/local/bin
which gdialog >/dev/null 2>&1 || install gdialog /usr/local/bin
grep -q "`cat calc.services`" /etc/services || cat calc.services >> /etc/services
install calc.xinetd /etc/xinetd.d/calc
ln -s /usr/local/bin/calc_ui /usr/local/bin/ncalc_ui
\end{code} % $
As we can see, there is a symbolic link from `\cmd{calc\_ui}' to `\cmd{ncalc\_ui}'.

OK. Let's test it:
\begin{code}{mverb}
$ ncalc_ui 
\end{code} % $
It works! But maybe we're wrong and this is just our old local calculator?
Let's check by stoping the `\cmd{xinetd}' service:
\begin{code}{mverb}
# systemctl stop xinetd
\end{code}
Our calculation ended with an error. And what about a non-networked `\cmd{calc\_ui}'?
\begin{code}{mverb}
$ calc_ui 
\end{code} % $
Still works. Let's run `\cmd{xinetd}' again:
\begin{code}{mverb}
# systemctl start xinetd
\end{code}
Networked `\cmd{ncalc\_ui}' works again! Quod erat demonstrandum~---
the networked version of the calculator created!
