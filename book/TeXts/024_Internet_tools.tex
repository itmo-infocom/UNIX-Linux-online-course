\section*{Internet Tools}

OK. But what about modern Internet world?

The main problem with these classic `\cmd{telnet}' and `\cmd{ftp}' tools is
insecurity: the user and password are transmitted over the network in
plain text and can be hijacked by an evil hacker. Today they have
practically been replaced by the \struct{Secure Shell} utilities. Secure Shell
provides secure, encrypted communication between untrusted hosts on
an unsecured network. And again we have a remote Shell and transfer tool:
\begin{itemize}
\item \cmd{ssh}~--- SSH client (remote login program)
\item \cmd{scp}~--- secure copy (remote file copy program)
\end{itemize}

\begin{code}{mverb}
man ssh
\end{code}
In \cmd{ssh}, you must specify the host and can set user and port. If you don't
set a user, by default you will try to log in with the same name as the local
user. Alternatively, you can add the command you want to run remotely directly
to the command line, without that ssh will just start an interactive shell
session on the remote host.

`\cmd{scp}' copies files between hosts on a network.  It uses the same
authentication and provides the same security as ssh, also data transfer
encrypted by ssh. You also may choose a port, you can use compression
while transferring.

Secure Shell utilities can be configured for passwordless authentication
using certificates.

\subsection*{Internet Data-transfer Utilities} % ## Internet data-transfer utilities

Finally, there are many tools that can be used to non-interactively
access network resources in scripts.

The first is the `\cmd{lynx}' text web browser. With the ``\cmd{-dump}''
parameter, it dumps the text formatted output of the URL of the WEB resource
to standard output. This output can then be used when processing the web page
in a script.

Also we have non-interactive network downloaders~--- `\cmd{wget}' and `\cmd{curl}'.
These tools can be used to download and mirror online resources offline.

`\cmd{lftp}'~--- sophisticated file transfer program with different access
methods~--- \struct{FTP}, \struct{FTPS}, \struct{HTTP}, \struct{HTTPS},
\struct{HFTP}, \struct{FISH}, \struct{SFTP} and \struct{file}.

And finally `\cmd{rsync}'~--- remote (and local) file-copying tool which reduces
the amount  of  data  sent over the network by sending only the differences
between the source files and the existing files in the destination.
\struct{Rsync} is widely used for backups and  mirroring and as an improved
copy command for everyday use. There  are two different ways for \cmd{rsync}
to contact a remote system: using a remote-shell program as the transport
(such as ssh or rsh) or contacting an rsync daemon directly via \struct{TCP}.
