\section*{Users and groups} % ## Users and groups

As we remember, users are one of the three pillars on which the UNIX world stands.

You can use some graphical interfaces to manage users and groups, but simple CLI utilities are often more convenient. There are:
\begin{itemize}
\item \cmd{adduser}, \cmd{useradd}~--- create a new user or update default
      new user information
\item \cmd{groupadd}~--- create a new group
\item \cmd{passwd}~--- update user’s authentication tokens
\end{itemize}

To create a new user, we (as `\struct{root}') simply have to run the program
`\cmd{adduser}' and set a password with `\cmd{passwd}'. But it's not over yet!

Actually, adduser is also black magic~--- in fact, all data related to users
and groups is placed in common text files that can only be modified
with ordinary text editors:
\begin{code}{mverb}
less /etc/passwd
\end{code}
The file format is quite simple~--- one line per user
with \struct{colon-separated fields}:
\begin{code}{mverb}
$ man 5 passwd
\end{code} % $
The fields, in order from left to right, are:

\begin{enumerate}
\item \struct{User name}: the string a user would type in when logging into
      the operating system: the logname. Must be unique across users listed
      in the file.
\item Information used to validate a user's \struct{password}; And at the very
      beginning, the password data was actually placed in this field.
      But we can read this file as a regular user, this is a design requirement.
      Did users have the ability to read passwords of other users at this time?
      Not. In Robert Morris and Ken Thompson's classic article
      \href{https://rist.tech.cornell.edu/6431papers/MorrisThompson1979.pdf}%
      {``Password Security: A Case History''}
%(https://rist.tech.cornell.edu/6431papers/MorrisThompson1979.pdf)
      about the UNIX password system, Morris described a real-life incident
      he himself saw:
      \begin{quote}
      Perhaps the most memorable such occasion occurred in the early 1960s when
      a system administrator on the CTSS system at MIT was editing the password
      file and another system administrator was editing the daily message that
      is printed on everyone’s terminal on login. Due to a software design
      error, the temporary editor files of the two users were interchanged and
      thus, for a time, the password file was printed on every terminal when it
      was logged in.
      \end{quote}
      And the main idea of UNIX passwords is not to believe that you can simply
      hide them. Better not to save passwords in the system at all. Actually,
      when creating a password, a random code was simply generated
      (the so-called \struct{SALT} code), and then from this code and password
      by means of a one-way `\struct{crypt}' procedure with the DES algorithm:
\begin{code}{mverb}
man crypt
\end{code}
      And the result of this operation cannot be decrypted (actually, we
      received some kind of hash)~--- when entering the system, the system
      receives SALT from the password field, encrypts it with the entered
      password, and simply compares it with the contents of the password field.

      In most modern uses, this field is usually set to ``\cmd{x}''
      (or ``\cmd{*}'', or some other indicator) with the actual password
      information being stored in a separate shadow password file. On Linux
      systems, setting this field to an asterisk (``\cmd{*}'') is a common way
      to disable direct logins to an account while still preserving its name,
      while another possible value is ``\cmd{*NP*}'' which indicates to use
      an NIS server to obtain the password.[2] Without password shadowing
      in effect, this field would typically contain a cryptographic hash of
      the user's password (in combination with a salt).
\item \struct{User identifier number}, used by the operating system for internal
      purposes. It need not be unique. Moreover, a superuser is simply a user
      with a zero user ID, and you can have multiple superusers in addition to
      the traditional ``\struct{root}'' superuser. For example, you can create
      some superuser with UID 0 and a name like ``\cmd{halt}'' and with
      the command ``\cmd{shutdown}'' as a shell for the user, and provide
      a password for that user to anyone who needs to shutdown the system
      at night.
\item \struct{Group identifier number}, which identifies the primary group of
      the user; all files that are created by this user may initially be
      accessible to this group. You can change this default during
      the current session with the command `\cmd{newgrp}'.
\item \struct{Gecos field}, commentary that describes the person or account.
      Some early Unix systems at Bell Labs used GE/Honeywell mainframe
      computers with General Comprehensive Operating System (GCOS) for print
      spooling and various other services, so this field was added to carry
      information on a user's GECOS identity.

      Typically, now this is a set of comma-separated values including
      the user's full name and contact details which may be used by some
      commands for example by mail user agent.
\item Path to the \struct{user's home directory}.
\item \struct{Program that is started} every time the user logs into the system.
      For an interactive user, this is usually one of the system's command line
      interpreters (shells). For example, for pseudo-users who do not need
      interactive sessions, this could be `\cmd{nologin}' or just `\cmd{false}'
      executables, which will exit immediately upon startup.
\end{enumerate}

The description of the groups is also placed in a text file:
\begin{code}{mverb}
less /etc/group
\end{code}
In this file, we see a similar format:
\begin{code}{mverb}
man 5 group
\end{code}
\begin{enumerate}
\item \struct{group\_name}~--- the name of the group.
\item \struct{password}~--- Password field that has never been used
\item \struct{GID}~--- the numeric group ID.
\item \struct{user\_list}~--- a list of the usernames that are members of
      this group, separated by commas.
\end{enumerate}

Finally, a file with real data regarding passwords:
\begin{code}{mverb}
ls -l /etc/shadow
\end{code}
As we can see, only the superuser has access to this file. The transfer of
password hashes from `\cmd{/etc/passwd}' to this file was carried out to
prevent brute-force attacks using modern computing equipment, which is now
becoming cheaper and cheaper. And we can see the hashes of the passwords in
the second field of the records for each user:
\begin{code}{mverb}
man 5 shadow
\end{code}
A password field which starts with an exclamation mark means that the password
is locked. The rest of the characters in the string represent the password field
before the password was locked, and you can simply remove the exclamation mark
to unlock it.

%[Under the hood -- special permission bits about `passwd` s-bit]
%(under_the_hood/special_permission_bits.md)

On a multiuser system with many administrators, it is advisable to use
the `\cmd{vipw}' and `\cmd{vigr}' commands to avoid conflicts when multiple
administrators are editing the same file at the same time:
\begin{code}{mverb}
man vipw
\end{code}
This file-based machinery for handling of user accounts is not hardcoded.
You can switch to network authentication services such as LDAP or Winbind
using the setup utility:
\begin{code}{mverb}
setup
\end{code}
or simply by editing the text configuration file `\cmd{/etc/nsswitch.conf}'
\begin{code}{mverb}
less /etc/nsswitch.conf
\end{code}
Other security related settings on Linux systems can be done in
the `\cmd{/etc/security}' and PAM configuration directories:
\begin{code}{mverb}
ls /etc/security/
ls /etc/pam.d/
\end{code}
As we can see, the UNIX system administration paradigm does not hide
the details from the user, everything can be configured by hands or scripts.
For beginners, such systems simply provide more or less user-firendly tools
and wizards to lower the barrier to entry.
