Let's add some networking to our design:
\begin{code}{mverb}
$ git checkout Example_6
Previous HEAD position was f266a24... GUI
HEAD is now at d3e5228... Network server
$ git diff Example_5 Example_6
...
$ ls
Makefile  README.md  calc  calc.services  calc.xinetd  calc_ui  gdialog
\end{code} % $
As you can see, just two new files are \cmd{calc.services} and \cmd{calc.xinetd}.
Because we will go the easy way~--- we will create a service for the \cmd{xinetd}
superserver. As we recall, for this we just need a program that reads standard
input and writes to standard output. And we also have such a program~---
this is our `\cmd{calc}' script!

To start our own network service, we just need to create the correct
configuration for the \cmd{xinetd} server and we have to extend our `\cmd{install}'
target in the Mnetcatakefile:
\begin{code}{mverb}
$ cat Makefile
...
\end{code} % $
We check the `\cmd{/etc/services}' configuration file for a `\cmd{calc}' service
port definition and if it doesn't exist, add it:
\begin{code}{mverb}
$ cat calc.services
calc            1234/tcp
\end{code} % $
As we can see, we are configuring our service on \struct{TCP port 1234}.

And finally we install calc xinetd config file a \cmd{/etc/xinetd.d/calc}:
\begin{code}{mverb}
$ sudo yum install xinetd
\end{code} % $
or in Ubuntu:
\begin{code}{mverb}
$ sudo apt install xinetd
\end{code} % $
and then:
\begin{code}{mverb}
$ sudo make install
install calc calc_ui /usr/local/bin
which gdialog >/dev/null 2>&1 || install gdialog /usr/local/bin
grep -q "`cat calc.services`" /etc/services || cat calc.services >> /etc/services
install calc.xinetd /etc/xinetd.d/calc
\end{code} % $
Let's restart `\cmd{xinetd}' after installing of our service:
\begin{code}{mverb}
$ sudo service xinetd restart
\end{code} % $
We can use `\cmd{telnet}' or the lighter `\cmd{netcat}' to test our server~---
this is the `\cmd{nc}' package on RH-like distributions:
\begin{code}{mverb}
$ sudo yum install nc
$ nc localhost 1234
Ncat: Connection refused.
\end{code}
or `\cmd{netcat}' on Ubuntu:
\begin{code}{mverb}
$ sudo apt install netcat
$ nc localhost 1234
$
\end{code} % $
Does not work\ldots Let's understand the problem~--- look in the system log.
As we recall, we can find it in `\cmd{/var/log/syslog}' on Debian-based Ubuntu
and `\cmd{/var/log/messages}' on RH-like systems.
And on any `\cmd{systemd}' based system we can use `\cmd{journalctl}' for this:
\begin{code}{mverb}
$ sudo journalctl
...
... xinetd[4449]: Server /data/home/sadov/works/courses/calc is not executable 
... xinetd[4449]: Error parsing attribute server - DISABLING SERVICE [file=/et
\end{code} % $
Well. We found the root of the problem: this is another mistake of mine~---
I did not change my experimental ``\cmd{calc}'' script path to our standard
``\cmd{/usr/local/bin/calc}''. Let's fix it:
\begin{code}{mverb}
$ sudo vim /etc/xinetd.d/calc
...
        server          = /usr/local/bin/calc
...
\end{code} % $
And restart `\cmd{xinetd}' service:
\begin{code}{mverb}
$ sudo service   restart
\end{code} % $
Redirecting to \cmd{/bin/systemctl} restart xinetd.service
\begin{code}{mverb}
$ nc localhost 1234
2+3
5
^C
\end{code} % $
Great - we're in the network now!
