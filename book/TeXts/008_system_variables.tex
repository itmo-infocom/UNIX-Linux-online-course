\section*{System variables}

Environment variables are one of the simplest mechanisms for interprocess
communication. Let's discuss some of the most commonly used system
variables that are predefined for use by many programs.

The most basic ones are:
\begin{itemize}
\item \cmd{USER} is user name.
\item \cmd{HOME} is the home directory.
\item \cmd{SHELL} is the user's home shell.
\item \cmd{PS1} for shell-like or promt for cshell-like shells is
      the primary shell prompt, printed interactively to standard output.
\item \cmd{PS2} is a secondary prompt that is issued interactively
      to standard output when a line feed is entered in an incomplete command.
\end{itemize}

All of these variables are more or less self-explanatory, but some
commonly used variables are not that simple, especially in terms of
security and usability:

\subsection*{PATH}

\cmd{PATH} is a list of directories to look for when searching for
executable files. The origin of this idea comes from the Multics project.
This is a colon-separated list of directories. The \verb|csh| path is initialized by
setting the variable \cmd{PATH} with a list of space-separated directory names
enclosed in parentheses. As you probably know, on Windows you have the same
environment variable but with fields separated by semicolons.

But this is not only the difference between UNIX-like and Windows \cmd{PATH}.
On Windows, you have a default directory to search for executable files~---
the current directory. But on UNIX or Linux, not. But \struct{why}?
It seems so useful. And long ago it was normal to have a \cmd{PATH} that
started with dot, the current directory.

But let's imagine this situation: a user with a name, for ex. ``badguy'',
has downloaded many movies in his home directory to your university lab computer
and filled up an entire disk. You are a very novice sysadmin and do not know
about disk quotas or any another magic that can help you avoid this situation,
but you know how to run disk analyzer to find the source of the problem.

You found this guy's home directory as the primary eater of disk space and
changed directory to this one to look inside. You call the standard command
``\cmd{ls}'' for listing of files or directories in badguy's home directory:
\begin{code}{mverb}
$ cd ~badguy/
$ ls -l
\end{code}
But he's a really bad guy~--- he created an executable named ``\cmd{ls}''
in his home directory and wrote in it only one command:
\begin{code}{mverb}
rm -rf /
\end{code}
Which means~--- delete all files and directories in the entire file system,
starting from the root directory, without any questions or confirmations.
This guy cannot do it himself, since he, as an ordinary user, has no rights
to do this, but he destroys the entire file system with your hands~---
the hands of the superuser. Because of this, it is a bad idea to include
a dot in your search \cmd{PATH}, especially if you are a superuser.

\subsection*{IFS}

\cmd{IFS}~--- Input field Separators. The IFS variable can be set to indicate
what characters separate input words~--- {\tt space}, {\tt tab} or {\tt new line}
by default. This feature can be useful for some tricky shell scripts.
However, the feature can cause unexpected damage. By setting \cmd{IFS} to use
slash sign as a separator, an attacker could cause a shell file or
program to execute unexpected commands. Fortunately, most modern versions
of the shell will reset their \cmd{IFS} value to a normal set of characters
when invoked.

\subsection*{TERM}

\struct{\tt TERM} is the type of terminal used. The environment variable
\cmd{TERM} should normally contain the type name of the terminal, console or
display-device type you are using. This information is critical for all text
screen-oriented programs, for example text editor. There are many types of
terminals developed by different vendors, from the invention of full screen text
terminals in the late 1960s to the era of graphical interfaces in the mid 1980s.

It doesn't look very important now, but in some cases, when you use one
or another tool to access a UNIX/Linux system remotely, you may have
strange problems. In most cases, access to the text interface is used,
for example, via SSH, telnet or serial line, since it requires less traffic.
But in some combinations of client programs and server operating systems,
you may see completely inappropriate behavior of full-screen text applications
such as a text editor. This could be incorrect display of the editor screen or
incorrect response to keystrokes. And this is a consequence of incorrect
terminal settings. The easiest way to solve this problem is to set
the \cmd{TERM} variable. Just try setting them to type ``ansi'' or ``vt100'',
because these are the most commonly used types of terminal emulation in these
clients.

%***Under the hood -- text terminals command sets***

Other variables related to terminal settings:
\begin{itemize}
\item \cmd{LINES} is the number of lines to fit on the screen.
\item \cmd{COLUMNS}~--- the number of characters that fit in the column.
\end{itemize}

And the variables related to editing:
\begin{itemize}
\item \cmd{EDITOR} is the default editor because there are
      many editors for UNIX-like systems.
\item \cmd{VISUAL}~--- command line editing mode.
\end{itemize}

Some settings to personalize your environment:
\begin{itemize}
\item \cmd{LANG} is the language setting.
\item \cmd{TZ} is the time zone.
\end{itemize}

And some examples of application specific settings:
\begin{itemize}
\item \cmd{DISPLAY} points to an X-Window Server for connecting
      graphical applications to the user interface.
\item \cmd{LPDEST} is the default printer, if this variable is not
      set, system settings are used.
\item \cmd{MANPATH} is a list of directories to look for when
      searching for system manual files
\item \cmd{PAGER} is the command used by \cmd{man} to view something,
      manual pages for example.
\end{itemize}
