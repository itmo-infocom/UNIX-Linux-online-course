Well. Let's go back to the fifth example:
\begin{code}{mverb}
$ git checkout Example_5
Previous HEAD position was d7f58c6...
Shell spec. symbols escaping and input formatting for 'expr'.
HEAD is now at f266a24... GUI
$ git diff d6453c0c41548a55e3249ea8c3b788c71cb76f7 Example_5 | less
...
\end{code}
Looks too long, but let's take a closer look~--- actaully added only
the gdialog file:
\begin{code}{mverb}
$ ls
Makefile  README.md  calc  calc_ui  gdialog
\end{code} % $
This is not our development~--- it's just a GTK+ graphical analogue of
the `\cmd{dialog}' text program. Our changes simply add an install command
for \struct{gdialog} if not already installed:
\begin{code}{mverb}
cat Makefile
\end{code}
and change four lines in `\cmd{calc\_ui}':
\begin{code}{mverb}
cat calc_ui
\end{code}
We first check for the existence of the `\cmd{gdialog}' program, and if it
exists on our system, we set the \cmd{DIALOG} environment variable as
`\cmd{gdialog}', if not, we set it as `\cmd{dialog}'. And then just we replaced
all the places where the `\cmd{dialog}' is used with the environment variable
\cmd{DIALOG}.

OK. Install the new version:
\begin{code}{mverb}
$ sudo make install
[sudo] password for liveuser: 
install calc calc_ui /usr/local/bin
which gdialog >/dev/null 2>&1 || install gdialog /usr/local/bin
\end{code} % $
and start our script again:
\begin{code}{mverb}
$ calc_ui
\end{code} % $
It works. But remember, the only thing used to communicate between the XWindow
server and the client is the \cmd{DISPLAY} environment variable:
\begin{code}{mverb}
$ echo $DISPLAY
:0
\end{code}
Let's unset it and try to run our UI again:
\begin{code}{mverb}
$ unset DISPLAY
$ calc_ui
\end{code}
Tadaam -- we've got a text interface again! Simply because the `\cmd{gdialog}'
script automagically switches to the text `\cmd{dialog}' if we are working in
text mode. Just set \cmd{DISPLAY} and we get the GUI again:
\begin{code}{mverb}
$ export DISPLAY=:0
$ calc_ui
\end{code}
