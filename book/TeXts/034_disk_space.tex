\section*{Disk space} % ## Disk space

Another important task with data in your file system is \struct{archiving} and
\struct{backing up}. It's wise to look into your filesystems first to analyze
disk usage. In case your file system is full on many systems, some graphical
disk analysis program will run and you can detect problems visually. But we can
also do this job using command line tools that can help you automate some of
the admin tasks.

The main tool for reporting file system disk space usage is the `\cmd{df}'
utility:
\begin{code}{mverb}
man df
\end{code}
The most useful option is ``\cmd{-h, --human-readable}'' for human readable
print sizes in kilobytes, megabytes, gigabytes.

For a more accurate analysis, you can use the `\cmd{du}' utility to estimate
the file space usage of directories and files:
\begin{code}{mverb}
man du
\end{code}
So, we can get the size of some directory:
\begin{code}{mverb}
# du -hs /tmp/
136K	/tmp/
\end{code}
And the most commonly used options are ``\cmd{-k}'', which displays sizes in
kilobytes, and ``\cmd{-x}'', which means that it will scan only this file system
and skip directories on other file systems. Let's take a look at an example of
using the command line tools to find the largest directories and files:
\begin{code}{mverb}
$ du -kx /tmp | sort -rn | less
\end{code} % $
We examine the directory `\cmd{/tmp}', perform a numeric sorting of the sizes of
directories and files, and redirect the result to the viewer `\cmd{less}'
for analysis.

And after finding the largest files and directories, we can clean up our file
system and before this archive and back up some data. The easiest way is
to simply copy using the `\cmd{cp -a}' command to some external drive, or
using `\cmd{scp -rC}' or `\cmd{rsync -avz}' to a remote host.

Also, using the `\cmd{cp}' or `\cmd{scp}' commands, you can copy any partition
or the entire disk, because for us they are just files. But a more efficient way
to do this is with the `\cmd{dd}' command:
\begin{code}{mverb}
man dd
\end{code}

By default, it just copies stdin to stdout, perhaps with some re-coding.
But the most interesting options for us are: `\cmd{if}', `\cmd{of}', `\cmd{bs}',
`\cmd{count}', `\cmd{seek}' and `\cmd{skip}'. With a combination of these
options, we can select the input and output files, choose the block size
to increase speed of I\&O, the number of blocks we want to copy, and seek/skip
on output/input. Thus, we can cut and paste any fragment from one device or
file to another.

We can also use the `\cmd{od}' command to low-level view of a file or device
in different formats:
\begin{code}{mverb}
man od
\end{code}
For example~--- to our hard drive:
\begin{code}{mverb}
# od -bc /dev/sda1 | less
\end{code}
