\section*{System manuals}
%by Oleg Sadov — last modified Aug 10, 2020 10:37 PM

The easiest way to get information about the use of a command is with
the \struct{-h} option or \struct{-\mbox{}-help} for GNU long options.

Also since its inception, UNIX has come with an extensive set of
documentation. Some information is often found in the \cmd{/usr/doc} or
\cmd{/usr/local/doc} or \struct{/usr/share/doc} directories as text files.

But the cornerstone of the Unix help system is the man command. And the man
in this case is not about gender~--- it is just an abbreviation for manual.
\begin{code}{mverb}
man man
\end{code}
The \cmd{man} command has been traditional on UNIX since its inception,
was created in the teletype era and still works great on all types of equipment.
And in the synopsis of the man command, we see two worlds,
two UNIX utility systems: BSD and SYSV.

And at first we can see the different command syntaxes:
\begin{code}{mverb}
SYSV~--- \cmd{man [-t] [-s section] name}
GNU, BSD~--- \cmd{man [-t] [section] name}
\end{code}

Parts of the man page are more or less the same:\\
\struct{NAME}, \struct{SYNOPSIS}, \struct{DESCRIPTIONS}, \struct{FILE},
\struct{SEE ALSO}, \struct{DIAGNOSTIC}, \struct{BUGS}

The \struct{minus S} option with some integer parameter of man command
in the SYSV variant denotes a section of real paper manuals supplied with
the OS by vendor. For GNU/BSD flavors, use only the section number.
Section numbers are also different.\\
%....\\
%Pres-n p.3\\
%....

Let's look for example to well known C-languge function `printf' manual page.
But
\begin{code}{mverb}
man printf
\end{code}
``\cmd{man printf}'' shows us the man page for the shell command, not
the C~function. To see the manual for the C printf function, we must run:
\begin{code}{mverb}
man 3 printf
\end{code}

To view a list of printf-related manual pages, we must run:\\
\cmd{whatis}~--- search the whatis database (created by makewhatis)
for complete words\\
and\\
`\cmd{man -k}' or `\cmd{apropos}'~--- search the whatis database for strings.

The databases should be created by the `\cmd{makewhatis}' program,
which is usually started at night by the cron service. If you have a freshly
installed system and want to run any command related to the whatis database,
you posibly need to start the \cmd{makewhatis} program manually.

OK~--- let's look to real man page:
\begin{code}{mverb}
zless /usr/share/man/man1/man.1.gz
\end{code}

\cmd{Troff}(short for ``typesetter roff'')/\struct{nroff}(newer ``roff'') is
an implementation of a text formatting program, traditional for UNIX systems,
using a plain text file with markup. It is ideologically based on
the RUNOFF MIT program, developed in 1964, and after a series of source code
losses and rewrites, a C-based implementation was re-implemented in 1975.
Under the name Troff, it was accepted for use on the UNIX system and, of course,
into the AT\&T patent department.

See -- \href{http://manpages.bsd.lv/history.html}{http://manpages.bsd.lv/history.html}

The main advantage of this tool was portability and the ability to generate
printouts for various devices, from a common ASCII printer to high-quality
typographic photosetters. Creating new technologies such as PostScript printers
simply adds the appropriate output drivers for the markup renderer.
Compared to the now better known WYSIWYG systems, such systems have better
portability between platforms and higher typing quality. Moreover, such systems
are more focused on the programmatic creation of printed documents
without human intervention.

Let's take a look at an example of manually rendering the man page that
was hidden under the hood of the man command:
\begin{code}{mverb}
$ zcat /usr/share/man/man1/man.1.gz | tbl | eqn -Tascii | \
                                                   nroff -man | less
\end{code} % $

It's just a software pipeline in which we \struct{unpack} the compressed TROFF
source, go through the TROFF \struct{preprocessor for tables} and
\struct{math equations}, pass a troff variant named `\struct{nroff}' to output
a text terminal, and finally pass a text pager/viewer named `\struct{less}'.

What is it viewer? In the TTY interface, the man command seems like
a good one~--- when you run it, you get paper manuals that you can combine
into a book, put on a shelf, and reread as needed. On a full-screen terminal~---
before, \cmd{Ctrl-S}(stop)/\cmd{Ctrl-Q}(repeat) was enough for viewing,
because at first the terminals were connected at low speed (9600 bits per
second for ex.), and now special programs were used~--- viewers.

And in fact, when you run the ``\cmd{man}'' command, you see just a viewer's
interface, in most cases it is ``\cmd{less}''. We will discuss viewers further,
but in a nutshell, the most commonly used of them the `\cmd{less}' handles
the \struct{UP/DOWN} keys normally, exiting a program~--- the `\struct{q}' key
means `quit'.

Well. It looks great, but the man-style documentation representation has
certain limitations: it is only a textual representation without any
useful functions invented after that time~--- for example, impossibility of
using hypertext links.

To improve it, the GNU Project has created an information system that
also works on all types of alphanumeric terminals, but with hypertext support.
For many GNU utilities, the corresponding help files are in info format,
and the man pages recommend that you refer to \cmd{info}.

Info has its own user interface and the best way to learn it is to simply
run the `\cmd{info info}' command. The internal source of ``info'' is the text
markup files in texinfo format. From such files are generated text files
for viewing on terminals, and also by the \TeX\ typesetting system
generated documentation for printing.

\struct{\TeX} is another typesetting system (or ``formatting system'') that was
developed and mostly written by Donald Knuth, released in 1978. \struct{\TeX}
and its \struct{\LaTeX} extension are very popular in the scientific world
as a means of typing complex mathematical formulas.

OK. But the inability to use graphic illustrations and any kind of multimedia
context remained relevant. And in the past, almost every commercial UNIX system
vendor created their own help system, which includes both hypertext support
and graphics for ex., and worked in the X~Window System.

But now with the advent of HTML (yet another text markup language), such
reference information began to be provided in this format directly on
the system or on WWW servers. Often these HTML pages or print-ready PDF versions
are simply generated from some content oriented markup such as DocBook XML.
%or LinuxDoc.
