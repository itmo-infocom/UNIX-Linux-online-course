\section*{Input/Output Redirection}

The standard design pattern for UNIX commands is to \struct{read information
from the standard input stream} (by default~--- the keyboard of the current
terminal), \struct{write to standard output} (by default~--- terminal screen),
and \struct{redirect errors to standard error stream} (also the terminal screen),
unless specified in the command parameters anything else.
These defaults settings can be changed by the shell.

The command ends with a sign ``\struct{greater than}'' (\struct{$>$}) and
the file name, means \struct{redirecting standard output to that file}.
The application code does not change, but the data it sends to the screen
will be placed in this file.

And the command ends with a ``\struct{less than}'' sign (\struct{$<$}) and
a file name, which means \struct{redirecting standard input to that file}.
All data that the application expects from the keyboard is read from the file.

\struct{Double} ``\struct{greater than}'' (\struct{>\hspace{-0.4ex}>}) means
\struct{appended to the output file}.

\struct{Number two} with ``\struct{greater than}'' means \struct{redirecting
standard error to a file}. By default, stderr also prints to the screen and in
this way we can separate this stream from stdin.

And finally, such a magic formula:
\begin{code}{mverb}
prog 2>&1
\end{code}
This means \struct{stdout} and \struct{stderr} are combined into one stream.
You may use it with other redirection, for example:
\begin{code}{mverb}
prog > file 2>&1
\end{code}
This means to redirect standard output to one ``\struct{file}'', both standard
output and standard error streams. But keep in mind~--- such combinations
are not equivalent:
\begin{code}{mverb}
prog > file 2>&1 \colorbox{red}{!=} prog 2>&1 > file
\end{code}
In the second case, you first concatenate the streams and then split again
by redirecting stdout to the selected file. In this case, only the stdout file
will be put into the file, stderr will be displayed on the screen.
The order of the redirection operations is important!

\href{under_the_hood/streams_numbers.md}{Under the hood~--- about streams numbers}\\

So the question is: what are we missing in terms of symmetry?
It's obvious~--- double ``less than'' sign (\struct{<\hspace{-0.4ex}<}).

And this combination also exists! But what can this combination mean?
Append something to standard input? But this is nonsense. Actually this
combination is used for the so-called ``\struct{Here-document}''.
\begin{code}{mverb}
prog <<END_LABEL
.....
END_LABEL
\end{code}
After the \struct{double} ``\struct{less than}'' some label is placed
(\struct{END\_LABEL} in our case) and all text from the next line
to \struct{END\_LABEL} is sent to the program's standard input,
as if from the keyboard.

Be careful, in some older shells this sequence of commands expects
exactly what you wrote. And if you just wrote a space before \struct{END\_LABEL}
for beauty, the shell will only wait for the same character string with a
leading space. And if this line is not found, the redirection from ``here
document'' continues to the end of file and may be the source of some
unclear errors.

And finally, the \struct{pipelines}. They are created with a pipe
symbol placed ``\cmd{|}'' between commands. This means \struct{connecting
the standard output from the first command to the standard input of
the second command}. After that, all the data that the first command by default
sending to the screen will be sent to the pipe, from which the second command
will be read as from the keyboard.

Programs designed in this redirection and pipelining paradigm are very
easy to implement and test, but such powerful interprocess communication
tools help us create very complex combinations of interacting programs.
For example as such:
\begin{code}{mverb}
prog1 args1... < file1 | prog2 args2... | ... | progN argsN... > file2
\end{code}

The first program receives data from the file by redirecting stdin, sends
the result of the work to the pipeline through stdout and after a long
way through the chain of filters in the end the last command sends the
results to stdout which is redirected to the result file.

\href{file://md}%
{\emph{Under the hood~--- differences in redirection/pipelining in Windows}}
