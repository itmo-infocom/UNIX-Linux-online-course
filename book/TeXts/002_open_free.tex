\section*{OPEN and FREE SOFTWARE}

Initially, when computer R\&D projects were mostly just university research,
they were open source and free of charge as a common scientific research result.
In addition, scientists were very interested in widespread dissemination of
this result, because their reputation directly depended on the prominence of
their scientific work.

Commercialization has changed this world to a closed and paid model.
Not completely closed, because standardization is very important for
government agencies and corporate consumers to protect investments and
prevent vendor locking. As a consequence, many \struct{open standards} have been
created by committees and organizations: \struct{POSIX}, \struct{ANSI}, \struct{ISO}.

And openness was a serious weapon in business competition. For example,
some well-known companies, including Bull, DEC, IBM, HP, Hitachi, Philips,
Siemens and others, created the Open Software Foundation (OSF) to fight
SUN and AT\&T during the so-called ``Unix wars''. The POSIX subsystem (which
was actually just a description of UNIX-like systems standards) was
included in Microsoft Windows NT because in the 1980s the US federal
government required compatibility with this open standard for government
contracts.

This \struct{openness} is very important because we get more compatible systems
from different vendors, ideally without undocumented features. As a result,
we have a computing infrastructure with higher efficiency and lower cost of
ownership.

But for some people, especially in the scientific world, this is not enough.
And in 1985, \struct{Richard Stallman} from MIT published the GNU Manifesto
where he announced the \struct{GNU Project}. The main goal of this project was
to create a UNIX-like OS with a full set of UNIX utilities from completely
free software. The \struct{Free Software Foundation} (FSF) was formed
to support these activities.

But what is this \struct{freedom} in the computer world and how is it different
with \struct{openness}? This difference is most accurately described in licenses.
In the proprietary world, the most widely used are so-called \struct{copyright
licenses}, which usually restrict certain user rights. Even with legal
access to the source code, the copyright holder can harass the consumer,
as we saw in the USL versus BSD or SCO versus IBM lawsuits.\\
\href{https://www.gnu.org/philosophy/free-software-for-freedom.en.html}%
{\url{https://www.gnu.org/philosophy/free-software-for-freedom.en.html}}

In contrast, the free software world uses copyleft licenses. The most famous
permissive licenses for free software were published in the late 1980s.
Two of them, named BSD and MIT~--- the educational institutions in which
they were created~--- look almost the same and give us the following basic rights:
\begin{itemize}
\item The \struct{freedom to run the program} as you wish, for any purpose
      (freedom 0).
\item The \struct{freedom to study} how the program works, and change it so
      it does your computing as you wish (freedom 1). Access to the source code
      is a precondition for this.
\item The \struct{freedom to redistribute copies} so you can help others
      (freedom 2).
\item The \struct{freedom to distribute copies of your modified versions}
      to others (freedom 3). By doing this you can give the whole community
      a chance to benefit from your changes. Access to the source code is
      a precondition for this.
\end{itemize}

The most famous projects released under these licenses are \struct{BSD Unix} and
the MIT \struct{X-Window} graphics system. Such licenses do not restrict
the use of closed derivative projects and their proprietarization.
To avoid this, the \struct{GNU General Public License} (\struct{GNU GPL} or
\struct{GPL}) was developed. One important limitation added to the fundamental
freedoms of this license (run, learn, share and modify software) is
the limitation for closing. Any derivative work must be distributed under
the same or equivalent license terms.

And the main difference between open-specification software and truly free
software is that we actually have a de facto standard, not a de jure standard.
In free and open source software, we have working reference implementations that
can be used as a basis for future development and cross-testing. % free of charge.

It's interesting, but this license does not completely restrict the creation of
proprietary closed applications using GPL licensed software. For example,
the OS kernel or shared libraries, simply because they are not included
directly in the proprietary application code.

We now have many free and open source licenses approved by
the Open Source Initiative:\\
\href{https://opensource.org/licenses}{\url{https://opensource.org/licenses}}

Another challenge for the free and open world is patent lawsuits.
For example, in 2007, Microsoft threatened to sue Linux companies like Red Hat
over patent violations. To solve this problem \struct{GPLv3} license has been
created, along with some activities such as the Open Invention Network (OIN),
which have a defensive patent pool and community of patent non-aggression
which enables freedom of action in Linux.
