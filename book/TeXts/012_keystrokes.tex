\section*{Keystrokes}

A few words about keyboard shortcuts. They are actually very useful
for command line work. Let's take a look at them:

\begin{comment}
\noindent
\struct{erase}: erase single character: [Backspace] or [Ctrl]-[H], [Delete]
or [Ctrl]-[?]\\
\struct{werase}: erase word: [Ctrl]-[W]\\
\struct{kill}: erase complete line: [Ctrl]-[U]. This can be very useful when
you enter something wrong on an invisible line, such as when entering
a password.\\
\struct{rprnt}: renew the output: [Ctrl]-[R]\\
\struct{intr}: [Ctrl]-[C] or [Delete]. Kill current process. In fact,
these strange settings for the [Delete] key were used by some older UNIX.
And many were very confused when, when trying to delete incorrectly entered
characters, they killed the executable application.\\
\struct{quit}: [Ctrl]-[\textbackslash]. Kill the current process, but with
a memory dump. Such a dump can be used to analyze the internal state of programs
by the debugger. It can be created in the system automatically during a program
crash, if you have configured your system accordingly,
or like this~--- by [Ctrl]-[\textbackslash] keystroke to analyze state,
for example, a frozen program.\\
\struct{stop}: [Ctrl]-[S]. Stop a current process.\\
\struct{start}: [Ctrl]-[Q]. Continue a previously paused process. And if
the program seems to be frozen, first try pressing [Ctrl]-[Q] to resume
the process. Perhaps you accidentally pressed [Ctrl]-[S].\\
\struct{eof}: [Ctrl]-[D]. End of file mark. Can be used to complete input of
something.\\
\struct{susp}: [Ctrl]-[Z]. As you probably know, this is the EOF mark on Windows
systems. But on UNIX-like systems, it stops the current process and disconnects
it from the current terminal line. After that, the execution of this process
can be continued in the foreground or in the background.
\end{comment}

\noindent
\begin{tabular}{lp{0.25\textwidth}p{0.65\textwidth}}
\struct{erase} &erase single character&\struct{[Ctrl]-[H]}\hspace{0.6em}or\hspace{0.6em}\struct{[Ctrl]-[?]}
(or \struct{[Backspace]}, \struct{[Delete]})\\
\struct{werase}&erase word &\struct{[Ctrl]-[W]}\\
\struct{kill}  &erase complete line&\struct{[Ctrl]-[U]}.\\
&\multicolumn{2}{p{0.90\textwidth}}{This can be very useful when
you enter something wrong on an invisible line, such as when entering
a password.}\\
\struct{rprnt} &renew the output&\struct{[Ctrl]-[R]}\\
\struct{intr}  &Kill current process.&\struct{[Ctrl]-[C]}\\
&\multicolumn{2}{p{0.90\textwidth}}{In fact, these strange settings for
the [Delete] key were used by some older UNIX. And many were very confused when,
when trying to delete incorrectly entered characters,
they killed the executable application.}\\
\struct{quit}  &Kill the current process with dump&\struct{[Ctrl]-[\textbackslash]}\\
&\multicolumn{2}{p{0.90\textwidth}}{Kill the current process, but with
a memory dump. Such a dump can be used to analyze the internal state of
programs by the debugger. It can be created in the system automatically during
a program crash, if you have configured your system accordingly, or like this --
by [Ctrl]-[\textbackslash] keystroke to analyze state, for example,
a frozen program.}\\
\struct{stop}  &Stop a current process&\struct{[Ctrl]-[S]}\\
\struct{start} &Continue a previously paused process&\struct{[Ctrl]-[Q]}\\
&\multicolumn{2}{p{0.90\textwidth}}{Continue a previously paused process.
And if the program seems to be frozen, first try pressing [Ctrl]-[Q] to resume
the process. Perhaps you accidentally pressed [Ctrl]-[S].}\\
\struct{eof}   &End of file mark&\struct{[Ctrl]-[D]}\\
&\multicolumn{2}{p{0.90\textwidth}}{Can be used to complete input of something.}\\
\struct{susp}  &stops the current process and disconnects
it from the current terminal line&\struct{[Ctrl]-[Z]}\\
&\multicolumn{2}{p{0.90\textwidth}}{As you probably know, this is the EOF mark
on Windows systems. But on UNIX-like systems, it stops the current process and
disconnects it from the current terminal line. After that, the execution of
this process can be continued in the foreground or in the background.}
\end{tabular}

\subsubsection*{KSH/Bash keyboard shortcuts.}

[ESC]-[ESC] or [Tab]: Auto-complete files and folder names.

This is very useful for dealing with UNIX-like file systems with very
deep hierarchical nesting. As we will see later, three levels of nesting
is a common place for such systems. Of course, we can use file management
interfaces like graphical file managers or text file managers like
Midnight Commander (mc), reminiscent of Norton Commander. But as we can
see, in most cases, the autocompletion mechanism makes navigating the
file system faster and can be easier if you know what you are looking for.

To use this machinery, you just need to start typing what you want
(command name, file path or environment variable name), press \struct{[Tab]}
and the shell will try to help you complete the word. If it finds an
unambiguous match, the shell will simply complete what it started. And if
we have many variants, Shell will print them and wait for new characters
to appear from us to unambiguously start the line. For example:
\begin{code}{mverb}
$ ec[tab]ho $TE[Tab]RM
xterm-256color
$ ls /u[tab]sr/l[Tab]
lib/ lib32/ lib64/ libexec/ libx32/ local/ 
o[Tab]cal/
bin etc games include lib man sbin share src
$
\end{code}

\noindent
\struct{\tt [Ctrl]-[P]}~--- Go to the previous command on ``history''\\
\struct{\tt [Ctrl]-[N]}~--- Go to the next command on ``history''\\
\struct{\tt [Ctrl]-[F]}~--- Move cursor forward one symbol\\
\struct{\tt [Ctrl]-[B]}~--- Move cursor backward one symbol\\
\struct{\tt [Meta]-[F]}~--- Move cursor forward one word\\
\struct{\tt [Meta]-[B]}~--- Move cursor backward one word\\
\struct{\tt [Ctrl]-[A]}~--- Go to the beginning of the line\\
\struct{\tt [Ctrl]-[E]}~--- Go to the end of the line\\
\struct{\tt [Ctrl]-[L]}~--- Clears the Screen, similar to the ``clear'' command\\
\struct{\tt [Ctrl]-[R]}~--- Let’s you search through previously used commands\\
\struct{\tt [Ctrl]-[K]}~--- Clear the line after the cursor

Looks more or less clear except for the Meta key. The Meta key was a modifier
key on certain keyboards, for example Sun Microsystems keyboards. And this key
used in other programs~--- Emacs text editor for ex. On keyboards that lack
a physical Meta key (common PC keyboard for ex.), its functionality may be
invoked by other keys such as the Alt key or Escape. But keep in mind the main
difference between such keys~--- the Alt key is also a key modifier and must be
pressed at the same time as the modified key, but ESC generally is
a real ASCII character (\struct{\mbox{27}/\mbox{hex 0x1B}/ \mbox{oct 033}}) and
is sent sequentially before the next key of the combination.

Another key point is that the origins of these key combinations are different.
The second is just the defaults for those specific shell and can be changed
using the shell settings. But the first one is the TTY driver settings.
And if we want to change such keyboard shortcuts, for example, so that
the Delete key does not interrupt the process, we can do this by asking
the OS kernel to change the parameter of the corresponding driver.
As we will see later, this can be done with the ``stty'' utility.
