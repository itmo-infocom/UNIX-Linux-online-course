Let's move on to the next tag:
\begin{code}{mverb}
$ git checkout Example_3
HEAD is now at d7f58c6... Shell spec. symbols escaping and input formatting for 'expr'.
$ git diff Example_2 Example_3
...
\end{code}
As you can see, the files have changed: \struct{README} and our script `\cmd{calc}'.
Let's take a look at `\cmd{calc}':
\begin{code}{mverb}
$ cat calc 
#!/bin/sh

FILE=/tmp/calc-$$

read EXPR
echo $EXPR | sed 's/\([^0-9]\)\([^0-9]\)/\1 \2/g; s/\([0-9]\+\)\([^0-9]\)/\1 \2/g; s/\([^0-9]\)\([0-9]\+\)/\1 \2/g; s/\([*()]\)/\\\1/g; s/^/expr /' > $FILE
sh $FILE
EXIT_STATUS=$?
rm -f $FILE
exit $EXIT_STATUS
\end{code} % $

It looks more complicated~--- let's analyze the changes. So the main idea
behind the fixes is to read the expression from stdin, not from the script
parameters, and modify it to avoid the Shell matching mechanism.

OK. On the third line, we define a \cmd{FILE} variable with a unique name
to store temporary data for evaluating `\cmd{expr}'. The uniqueness is
guaranteed by a special double dollar environment variable that indicates
the PID of the current process.

On the fifth line, we read the expression from stdin into the \cmd{EXPR}
variable. We then edit it on the fly with the `\cmd{sed}' stream editor.
With the `\cmd{sed}' command, we insert spaces between numbers and signs of
arithmetic operations, we do the escaping with the slash character before
the asterisk and brackets. Finally, before the expression, insert
the `\cmd{expr}' command and redirect the output from `\cmd{sed}' to
a temporary file defined at the beginning of the script.

We then run our temporary script with `\cmd{sh}', getting the exit status from
the special variable ``\struct{dollar question}'' into ``\cmd{EXIT\_STATUS}''
variable, deleting the temporary file, and exiting with the parameters stored in
``\cmd{EXIT\_STATUS}''. We need such a complex construction because
`\cmd{rm -rf}' will return a success status regardless of the result of
evaluating the expression.

OK~--- let's check our fixes:
\begin{code}{mverb}
$ ./calc
2*2
4
\end{code} % $
It works! And we don't even need to insert spaces between parts of our
expression. Let's check~--- maybe something else was broken by our corrections?
\begin{code}{mverb}
$ ./calc
1+2
3
$ ./calc
5-7
-2
$ ./calc
6/3
2
$ ./calc
7/3
2
$ ./calc
7%3
1
\end{code} % $
Looks good. What about expected errors?
\begin{code}{mverb}
$ ./calc
6/0
expr: division by zero
$ echo $?
2
\end{code} % $
Great~--- we got it!

In fact, we can implement a simpler solution and in the sixth line just write:
\begin{code}{mverb}
echo $(($EXPR)) > > $FILE
\end{code} % $

But we wrote a script that still works with older shells that may not support
such constructs, and we also looked at how to use non-interactive editing
in scripts.
