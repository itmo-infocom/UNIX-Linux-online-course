\section*{Command Interpreter}

The first characters that you can see at the beginning of a line when you log
in and access the command line interface is the so-called \struct{Shell prompt}.
This prompt asks you to enter the commands. In the simplest case,
in the Bourne shell, it's just a \struct{dollar sign} for a regular user and
a \struct{hash sign} for a superuser. In modern shells, this can be a complex
construct with host and user names, current directory, and so on.
But the meaning of the dollar and hash signs is still the same.

The Shell as a command interpreter that provides a compact and powerful
means of interacting with the kernel and OS utilities. Despite the many
powerful graphical interfaces provided on UNIX-like systems, the command
line is still the most important communication channel for interacting
with them.

All commands typed on a line can be used in command files executed by
the shell, and vice versa. Actions performed in the command interpreter can
then be surrounded by a graphical UI, and the details of their execution,
thus, will be hidden from the end user.

Each time a user logs into the system, he finds himself in the environment of
the so-called home interpreter of the user, which performs configuration
actions for the user session and subsequently carries out interactive
communication with the user. Leaving the user session ends the work of
the interpreter and processes spawned from it. Any user can be assigned any of
the interpreters existing in the system, or an interpreter of his own design.
At the moment, there is a whole set of command interpreters that can be a user
shell and a command files executor:
\begin{itemize}
\item \cmd{sh} is the \struct{Bourne-Shell}, the historical and conceptual
      ancestor of all other shells, developed by Stephen Bourne at AT\&T Bell
      Labs and included as the default shell for Version 7 of Unix.
\item \cmd{csh}~--- \struct{C-Shell}, an interpreter developed at the University
      of Berkeley by Bill Joy for the 3BSD system with a C-like control
      statement syntax. It has advanced interactive tools, task management tools,
      but the command file syntax was different from Bourne-Shell.
\item \cmd{ksh}~--- \struct{Korn-Shell}, an interpreter developed by David Korn
      and comes standard with SYSV. Compatible with Bourne-Shell, includes
      command line editing tools. The toolkit provided by Korn-Shell has been
      fixed as a command language standard in POSIX P1003.2.
\end{itemize}

In addition to the above shells that were standardly supplied with
commercial systems, a number of interpreters were developed,
which are freely distributed in source codes:
\begin{itemize}
\item \cmd{bash}~--- \struct{Bourne-Again-Shell}, quite compatible with
      Bourne-Shell, including both interactive tools offered in C-Shell and
      command line editing.
\item \cmd{tcsh}~--- \struct{Tenex-C-Shell}, a further development of
      the C-Shell with an extended interactive interface and slightly improved
      scripting machinery.
\item \cmd{zsh}~--- \struct{Z-Shell}, includes all the developments of Bourne-Again-Shell
      and Tenex-C-Shell, as well as some of their significant extensions
      (however, not as popular as bash and tcsh).
\item \cmd{pdksh}~--- \struct{Public-Domain-Korn-Shell}, freely redistributable
      analogue of Korn-Shell with some additions.
\end{itemize}

There are many ``small'' shells often used in embedded or mobile systems
such as \struct{ash}, \struct{dash}, \struct{busybox}.
