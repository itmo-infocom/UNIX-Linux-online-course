OK, let's take a step forward:
\begin{code}{mverb}
$ git checkout Example_2
Previous HEAD position was 8a16efc... Added simple Makefile
HEAD is now at 757a731... Added "calc" shell-script.
\end{code} % $
We see some changes in the README:
\begin{code}{mverb}
$ cat README.md 
...
\end{code} % $
and in our repository~--- we can see the `\cmd{calc}' script:
\begin{code}{mverb}
$ ls
calc  Makefile  README.md
\end{code} % $
This means~--- we can mark some stages of our development with tags.
And then, using the ``\cmd{checkout}'' operation, we can switch between them
at any time. OK~--- let's look at our calculator:
\begin{code}{mverb}
$ cat calc
#!/bin/sh

expr $*
\end{code} % $
Wow~--- looks pretty simple! We simply call the expr program with the arguments
passed to our script. And as we can see, `\cmd{expr}' is just an expression
evaluator:
\begin{code}{mverb}
man expr
\end{code} % $
And this is the usual way of UNIX development~--- not reinvent the wheel,
but just take parts of them and glue them with a Shell. OK~--- let's try
to test:
\begin{code}{mverb}
$ ./calc 1 + 2
3
\end{code} % $
We understand that the expression is just the arguments of the `\cmd{expr}'
command and we must separate them with spaces.
\begin{code}{mverb}
$ ./calc 5 - 7
-2
$ ./calc 6 / 3
2
\end{code}
Looks good~--- let's try divide:
\begin{code}{mverb}
$ ./calc 6 / 3
2
$ ./calc 6 / 0
expr: division by zero
$ ./calc 7 / 3
2
\end{code} % $
OK, got it~--- `\cmd{expr}' performs integer operations and we have to use
another operation for the remainder of the division:
\begin{code}{mverb}
$ ./calc 7 % 3
1
\end{code} % $
But:
\begin{code}{mverb}
$ ./calc 2 * 2
expr: syntax error
\end{code} % $
What's wrong? Let's try to debug:
\begin{code}{mverb}
$ sh -x ./calc 2 * 2
+ expr 2 Makefile README.md calc 2
expr: syntax error
\end{code} % $
Oh yeah~--- the asterisk are just special Shell matching characters,
in this case meaning~--- all files in the current directory!
And it is replaced by all files from the current directory.
We need to fix it.
