\section*{Network configuration} % ## Network configuration

Finally, let's discuss the administrative tasks associated with the network.
In most cases, after installing the system, you have a more or less configured
network in accordance with the \struct{DHCP} settings of your local network.
The most you need is to set a password for your \struct{WiFi}.

But in some cases it would be helpful to have some utilities to view and
manage network settings. Unlike other devices, network interfaces do not
have their own representations in the device files in the \cmd{/dev} directory.
You can work with them only with the help of special utilities.
Traditional set of utilities for network configuration:
\begin{itemize}
\item \cmd{ifconfig}~--- configure a network interface
\item \cmd{route}~-- show/manipulate the IP routing table
\end{itemize}

With no arguments, \cmd{ifconfig} just shows us the current state of enabled
network interfaces. By pointing to a network interface, we can ``\struct{up}''
and ``\struct{down}'' them, we can also manually set the \struct{IP address}
and \struct{netmask}:
\begin{code}{mverb}
man ifconfig
\end{code}

You can use the `\cmd{route}' utility to work with the routing table.
To view the route table we may run command:
\begin{code}{mverb}
route -n
\end{code}
The option `\cmd{-n}' show  numerical  addresses instead of trying to determine
symbolic host names. This can be useful if you are having problems accessing
the DNS server.
\begin{code}{mverb}
man route
\end{code}
With this command we can add and remove routes to hosts and networks.
We can set gateways to them.

In modern Linux distributions, these `\cmd{net-tools}' are outdated and are not
installed by default. They are migrating to the more versatile `\cmd{ip}'
utility, which also supports more advanced networking options than
`\cmd{net-tools}':
\begin{code}{mverb}
man ip
\end{code}
You can use `\cmd{ip link}' to perform the same tasks as `\cmd{ifconfig}' and
`\cmd{ip route}' to replace `\cmd{route}':
\begin{code}{mverb}
$ ip link help 2>&1 | less
$ ip route help 2>&1 | less
\end{code}
The next important task when setting up your network is setting up \struct{DNS
resolving}. This configuration is placed in `\cmd{/etc/resolv.conf}':
\begin{code}{mverb}
$ cat /etc/resolv.conf
\end{code} % $
Here we can configure up to 3 nameservers.
\begin{code}{mverb}
man resolv.conf
\end{code}
Also, using the `\cmd{search}' configuration option, we can configure
the domains in which short names will be searched.

\subsection*{Network access} % ## Network access

And finally, a few words about regulating network access to your computer.
As we understand, network attacks are currently the most dangerous.
And the most famous tool for restricting access is a firewall.

At its deep level, the firewall in Linux is controlled by the `\cmd{iptables}'
utility and moving to `\cmd{nftables}'. But at a higher level, different
systems manage it in different distributions:
\begin{itemize}
\item Canonical’s Uncomplicated Firewall (`\cmd{ufw}') to configure the iptables
      on every Ubuntu and Debian system I’ve used in recent years.
      The firewall isn’t enabled by default in Ubuntu nor installed by default
      in Debian. As its name suggests, it’s fairly uncomplicated to set up and
      maintain. It has an easy to remember syntax that’s more friendly to human
      users than the underlying iptables rules.
\item Fedora and Red Hat Enterprise Linux enables \struct{FirewallD} by default.
      Its `\cmd{firewall-cmd}' front-end has almost the same feature set for
      basic firewall management as `\cmd{ufw}', and adds network zone management
      to the mix. Zone management allows you to set up presets with rules for
      different network conditions/locations. For example ``\struct{Home}'' and
      ``\struct{Office}'' where all communications with local machines are allowed,
      and ``\struct{Public Wi-Fi}'' where no communication with the same subnet
      would be allowed. Rules can be applied automatically per network interface,
      or used through \struct{NetworkManager} and the \struct{GNOME network GUI}
      `\cmd{system-config-firewall}'.
\end{itemize}

Both front-ends come with pre-defined rules for common server services and
protocols. These rules include a keyword/name and associated industry standards
and default ports for running services publicly. They each come in their own
formats that aren’t interoperable with each other, of course. \cmd{ufw} uses
service-named files containing one line with port and protocol, and
\struct{FirewallD} uses six lines of XML to create the same profile.

The best policy is to simply close all services from the Internet that
you do not need on your computer, and only open those that you need
for remote access. For example~--- \struct{SSH}.

And then just use the lighter tweak tool~--- host access control files:
\begin{code}{mverb}
/etc/hosts.deny:
          ALL: ALL
/etc/hosts.allow:
          ALL: LOCAL @some_netgroup
          ALL: .foo.bar EXCEPT hacker.foo.bar
\end{code}
This is the configuration for the so-called \struct{tcp wrapper}, which was
originally developed as a `\cmd{tcpd}' service for the `\cmd{inetd}' superserver,
and now its functionality is included in the `\cmd{libwrap}' library
that is used by several network services such as NFS.

Again, the easiest way is to disable everything in the configuration file
`\cmd{/etc/hosts.deny}' and allow only a fixed list of hosts to access your
computer, for example, via SSH in `\cmd{/etc/hosts.allow}'. Using this
mechanism is easier than using a firewall because all changes are made
immediately after saving the configuration file~--- you don't need to
reconfigure and reload any services.

\subsection*{Network services} % ## Network services

And if we talk about older classical systems, then tcp\_wrapper `\cmd{tcpd}' is
configured through the `\cmd{inetd}' service. As we'll see, this is a so-called
superserver designed to make the life of network developers easier. To create
the server side of a network application, you just need to develop
an application that reads stdin and writes to stdout, and write the service
configuration to the `\cmd{inetd}' configuration file. All the work of
listening on network sockets and maintaining the connection makes
a super-server for you.

Classic services such as \struct{FTP}, \struct{POP3}, and \struct{telnet} were
designed to work with `\cmd{inetd}'. It is also possible to configure
an \struct{HTTP} server for access via `\cmd{inetd}'. Newer systems have
replaced `\cmd{inetd}' with `\cmd{xinetd}', which provides better protection
against DOS attacks, and replaced with `\cmd{systemd}' in newest systems.
