\section*{Text Editors}

OK. We can create a file using the `\cmd{cat}' utility and view the file using
a viewer. But what if we need to change something, especially if we only have
a TTY interface? And it is possible~--- we have a so-called line editor
named `\cmd{ed}'. The only interface for such an editor is the command line:
%[Pr-n 6 slide 5]
(\url{http://sdn.ifmo.ru/education/courses/free-libre-and-open-source-software/lectures/lecture-6}).

So let's try to edit new file.
\begin{code}{mverb}
$ ed tst
tst: No such file or directory
\end{code} % $
At first~--- we will add some lines:
\begin{code}{mverb}
i
1 2
3 4
\end{code}
and we must end our input with one `dot' per line.
\begin{code}{mverb}
.
\end{code}
Let's take a look at our file, moving to the first line:
\begin{code}{mverb}
1
1 2

3 4

?
\end{code}
Seems good. Now we can add something to the end:
\begin{code}{mverb}
a
5 6
.
1
1 2

3 4

5 6

?
\end{code}
OK~--- we have 3 lines in the file now.
And finally~--- let's try to make a magic pass:
\begin{code}{mverb}
1,$s/\(.\) \(.\)/\2 \1/
1
2 1

4 3

6 5

?
\end{code} % $
This means: from the first to the last line, replace the lines where we have
two separate letters separated by a space, exchanging those letters with places.
And now `write' and `quit':
\begin{code}{mverb}
w
12
q
\end{code}
Let's check the result:
\begin{code}{mverb}
$ cat tst
2 1
4 3
6 5
\end{code} % $

But for what purposes can you use a line editor now that we have full-screen
editors with a user-friendly interface? Of course, you can imagine a situation
where your full-screen environment is broken and only the line editor will be
the salvation. And in general I had such situations. But the main use case
for \cmd{ed} is for automatic editing in scripts. Anything you need to change
in the text data can be done with this editor, including sophisticated regex
search and replace.

Moreover, we have a `\cmd{sed}'~--- stream editor, for editing text data
in pipelines:
\begin{code}{mverb}
$ sed 's/\(.\) \(.\)/\2 \1/' < tst
1 2
3 4
5 6
\end{code} % $

As you can see, the original file does not change, all changes are simply
sent to standard output:
\begin{code}{mverb}
$ cat tst
2 1
4 3
6 5
\end{code} % $

But UNIX-like systems also have full-screen editors, which can also be
confusing for beginners. It was developed by Berkeley student Bill Joy
for BSD initially as a visual mode for a line editor. It is a very fast
and lightweight editor that is part of the  Single Unix Specification and
the POSIX, which found on every UNIX-like system. The \cmd{VI} editor works
on all types of terminals and generally requires only a conventional letter
keyboard. You can work with it without the arrow keys, PgUp/Down or
anything similar. There are actually very small keyboards out there that
are optimized for `\cmd{vi}'.

But to work on it, you need to understand the basic concept of this editor:
it can be in three states
%[Pr-n 6 slide 6]
(\url{http://sdn.ifmo.ru/education/courses/free-libre-and-open-source-software/lectures/lecture-6}).

Immediately after launch, we find ourselves in the usual \struct{command mode}
and can switch to \struct{editing mode}, for example, by pressing
the ``{\tt [Insert]}'' key. As we can see, the mode status in the lower left
corner has changed to `-\mbox{}- INSERT -\mbox{}-', and now we can edit our file.
Pressing {\tt Insert} again will change the state mode to
`-\mbox{}- REPLACE -\mbox{}-' and vice versa. Exit the editing mode by pressing
{\tt ESC}. The third mode can be accessed by pressing the colon key in command
mode. This is \struct{`ed' mode}. In this mode, we can use the normal
`\cmd{ed}' line editor commands and finish them with {\tt ENTER}.

In command mode, you can find something with regex by slash and question marks,
as in the `\cmd{less}' viewer. In improved VI (\cmd{vim}), you can use very
useful visual mode by pressing {\tt V}. After that you can delete the selection
with `\cmd{d}' or just copy it with `{\cmd{y}' (yank). Then you can paste it
anywhere with `\cmd{p}' (paste). Moreover, you can use {\tt [Ctrl-V]} to select
a vertical box. To exit visual mode, simply press {\tt ESC}.

Also you may look to \cmd{vimtutor}~--- a guide to Vim can be useful
for beginners.

And the second most common editor is \cmd{Emacs}. This Richard Stallman's editor
was the starting point for the GNU Project, along with GCC and UNIX utilities.
EMACS means, for example, ``Editing MACroS''. An apocryphal hacker koan alleges
that the program was named after Emack\&Bolio's, a popular Cambridge ice cream
store. But overall it is a really very powerful editor with macro extensions,
allowing the user to override any keystrokes to launch the editor program.
But unlike other editors that support macro-extensions, in Emacs they are
implemented using the LISP programming language embedded on editor.
At the time, LISP was very popular in artificial intelligence in
the United States, and Stallman, who worked at the MIT Artificial Intelligence
Lab, chose it as the editor extension language.

This implementation allows many LISP-based applications to be developed,
including a user-friendly interface for developers in various programming
languages. Usually Emacs is a text editor with a simple graphical interface.
But it can only be run in a text environment. The most commonly used
keystrokes are:\\
\verb|C-c C-x|~--- exit\\
\verb|C-h t|~--- tutorial\\
\verb|C-h i|~--- info

\medskip
If you feel overwhelmed by the difficulty of Emaccs, you can see a personal
psychoanalyst: \cmd{M-x doctor}. It would spoil the fun and hurt your recovery
to say too much here about how the doctor works. But when you're ready,
you may try to find the well-known Turing-test related AI program ELIZA
on WikiPedia.

Also in the UNIX/Linux world, there are many other editors that may be more
convenient for you, such as:
\begin{itemize}
\item \cmd{joe}, \cmd{nano}~--- simple text editors or
\item \cmd{gedit}, \cmd{kate} -- text editors from Gnome and KDE projects
\end{itemize}
