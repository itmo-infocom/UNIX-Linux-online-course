\documentclass[12pt]{report}
\usepackage{cmap}

\usepackage[T1]{fontenc}
\usepackage[utf8]{inputenc}

\usepackage[english]{babel}
%\usepackage[english,russian]{babel}

\usepackage{hyperref}

\usepackage{amsthm}
\usepackage{amsmath}
\usepackage{amssymb}
\usepackage{amsfonts}
\usepackage{amsthm}
\usepackage{enumerate}
%\usepackage{enumitem}
\usepackage{tikz}
\usepackage{xcolor}
\usepackage{textcomp}

\usepackage{listings}

\usepackage{comment}

\newcommand{\struct}[1]{\textcolor{blue}{#1}}
\newcommand{\Rim}[1]{\uppercase\expandafter{\romannumeral#1}}
\newcommand{\cmd}[1]{\textcolor{blue}{\tt #1}}

\definecolor{mverb}{HTML}{C0E8F8}

\lstnewenvironment{code}[1]{%
  \lstset{backgroundcolor=\color{#1},
  tabsize=10,
  frame=single,
  showlines=true,
  framerule=0pt,
  basicstyle=\ttfamily,
  columns=fullflexible}}{}

\parskip5pt
\parindent0pt

\title% (optional, use only with long paper titles)
{UNIX AND LINIX\\IN INFOCOMMUNICATION\\Week 2}

\author % (optional, use only with lots of authors)
{O.~Sadov%,\\ \texttt{tit@astro.spbu.ru}
}
% - Give the names in the same order as the appear in the paper.
% - Use the \inst{?} command only if the authors have different
%   affiliation.

%%%\institute{ITMO}
%  Университет Информационных Технологий, Механики и Оптики\\
%  кафедра Телекоммуникационных Систем

% - Use the \inst command only if there are several affiliations.
% - Keep it simple, no one is interested in your street address.

\date% (optional, should be abbreviation of conference name)
{}
%{Aug 23, 2020 10:33 PM}


\begin{document}
\maketitle
\section*{Components}

The main design principle used in UNIX-like systems is the ``\struct{KISS}''
principle. KISS, an acronym for ``keep it stupid simple'' or more officially
``keep it short and simple'', is a design principle noted by the U.S. Navy
in 1960. The KISS principle states that most systems work best if they are kept
simple rather than made complicated; therefore, simplicity should be a key goal
in design, and unnecessary complexity should be avoided.

And from the very beginning UNIX was a very flexible modular system.
The basic set of components from which UNIX-like systems are built is:
\begin{itemize}
\item \struct{Kernel}
\item \struct{Shell}
\item \struct{Utilities}
\item \struct{Libraries}
\end{itemize}

The \struct{kernel} is the first bunch of OS code that is loaded onto your
computer's memory and run for execution. This program launches all
processes on the system, handles interactions between system resources,
and still live while your system is running. The kernel runs with
the highest privileges and has access to all system resources. All processes
in the system operate in user space and interact with such resources and
among themselves, sending requests to the kernel using special software
functions named ``\struct{system calls}''. And the kernel handles such requests
according to the permissions between the processes and resources.

\href{under_the_hood/interrupts.md}%
{Under the hood~--- kernel as a set of interrupt handlers}

But if we have a kernel, it seems reasonable to have a shell around it.
And we have this one. Oh, sorry, not one -- now there are many shells
dating back to the first Unix shell by Ken Thompson, introduced in 1971.
Actually, the \struct{shell} is the first and most important communicator between
human, programs, and kernel. Generally it's just a program that is
launched when the user logs in. It listens for standard input (usually
from the keyboard) and sends the output of commands to standard output
(usually to the screen).

The best-known implementation of the UNIX shell is the \struct{Bourn shell},
developed by Stephen Bourne at Bell Labs in 1979 and included as the
default interpreter for the Unix version 7 release distributed to
colleges and universities. Supported environment variables, redirecting
input/output streams, program pipes and powerful scripting. All modern
shells (and not only UNIX shells) inherit these capabilities from this
implementation.

The shell is a very effective glue for \struct{utilities} in multitasking
operating systems. The most commonly used utilities were developed early
in the life of UNIX. There are tools for working with users, groups, files,
processes. Since UNIX was originally created to automate the work of
the patent department, it has a rich set of tools for processing text files
and streams. The most commonly used design pattern for UNIX utilities is
the filter between standard input and standard output. An arbitrary
number of such programs can be combined into a so-called software
pipeline that uses Shell program pipes for interprocess communication.

Each such utility can be very simple, but together they can be a very
powerful compound program that fits on a single command line. Doug
McIlroy, head of the Bell Labs Computing Sciences Research Center, and
the inventor of Unix pipes, described the Unix philosophy as follows:
``\struct{Write programs that do one thing and do it well. Write programs
to work together. Write programs to handle text streams, because that is
a universal interface.}''

Currently, the most commonly used utilities are those of the GNU project,
which were created after the commercialization of UNIX. In most cases,
they are richer in their capabilities and more complex parameters than
the classic SYSV set of utilities that you see in commercial UNIXes.

On small embedded systems, you might see a systems like ``\struct{busybox}''
that looks like a single binary with many faces built from a configurable
modular library. It may look like a fully featured set of UNIX-style
utilities, including a shell and a text editor. And during the build phase,
you can choose exactly the features you need to reduce the size of
the application.

All utilities and shells are built on top of \struct{software libraries}.
They can be dynamically or statically linked. In the first case,
we have more flexibility for updates and customization and we get a set of
applications that take up less disk/memory space in total. In the second case,
we have a solid state application that is less dependent on the overall system
configuration and can be more platform independent. And as I said earlier
in the first case, you can use libraries with ``viral licenses'' (like the GPL)
to write proprietary applications, but in the second case, you cannot.

\href{under_the_hood/virtual_memory.md}% ***Under the hood -- MMU and shared libraries ***
{Under the Hood~--- Virtual Memory, MMU and shared libraries}

\section*{Your system}

The best advantage of free systems is their \struct{availability}. You can
download many kinds of Linux or UNIX systems for free. You can distribute
such systems, downloaded from the Internet, and use them to create your
own customized solutions using a huge number of already existing
components.

For example, for educational purposes we use our own Linux distribution
called \struct{NauLinux}. This title is the abbreviated title of the Russian
translation of the original \struct{Scientific Linux} project~---
``Nauchnyi Linux''. We are adding many software packages for working
software-defined networks and quantum cryptography emulating and are using
this new distributions to simulate various complex systems in educational or
research projects. This distribution is free and is used by students to
create their own solutions.

\struct{Scientific Linux} on which we are based is also free. It was created at
Fermilab for use in high energy physics and has focused from the beginning
on creating specialized flavours optimized for the needs of laboratories
and universities.

This, in turn, is based on \struct{Red Hat Enterprise Linux} (RHEL),
a commercial Linux distribution widely used in the industry. You will be
charged for using the binaries for that distribution and getting support
from the vendor. But the sources from this distro are still free, and anyone
can download it and rebuild their own distro.

The source of RHEL, in turn, is the \struct{Fedora experimental project},
developed by the community with the support of Red Hat. Leveraging large
communities of skilled and motivated users lowers the cost of testing,
development, and support for the company. And this is an example of
the profitable use of the Free and Open Source model by a commercial company.

There are many free BSD OS variants, currently \struct{FreeBSD}, \struct{NetBSD},
\struct{OpenBSD}, \struct{DragonFly BSD}. They all have their own specifics and
their own kernels with incompatible system calls. This is a consequence of
the fact that the development of these systems is driven by communities in
which disagreements arise from time to time, and they are divided according
to different interests regarding the development of systems.

On the Linux project, we still have one and only one benevolent dictator,
\struct{Linus Torvalds}. As a result, we still have a single main kernel
development thread published on \url{kernel.org}. While many other experimental
Linux kernel flavours are also being developed, not all of them are
accepted into the mainstream. In turn, on the basis of this single
kernel, the development of various OS distributions is made, often with
some changes from the distribution vendors.

The most commonly used free Linux distributions are:
\begin{itemize}
\item community driven \struct{Debian} project
\item \struct{Ubuntu} Ditro based on Debian and developing by Canonical company
\item The \struct{Fedora} Project on which RHEL development is based
\item and \struct{Centos}~--- another free RHEL respin
\end{itemize}
\struct{Gentoo}, \struct{Arch}, \struct{Alpine} and many other Linux
distributions are also well known. Many projects are focused on embedded systems,
such as BuildRoot, Bitbake, OpenWRT, OpenEmbedded and others.

You can usually use them in different ways - install on your computer
(including coexistence with other systems such as Windows, with the ability
to dual boot), boot from a live image or network server without installing
the OS to a local hard drive, run as a container or virtual machine on your
local computer or network cloud, etc. You can find detailed information
on installing and configuring such systems in the documentation of
the respective projects.

Moreover, you can simply use online services to access UNIX or Linux
systems, for example:
\begin{itemize}
\item \href{https://skn.noip.me/pdp11/pdp11.html}%
      {\url{https://skn.noip.me/pdp11/pdp11.html}}~--- PDP-11 emulator with UNIX
\item \href{https://bellard.org/jslinux}{\url{https://bellard.org/jslinux}}~---
      a lot of online Linux'es
\end{itemize}

When you log in, you will be asked for a user and password. Depending on
your system configuration, after logging in, you will have access to
a graphical interface or text console. In both cases, you will have access
to the \struct{Shell command line} interface, which we are most interested in.

The user password is set by the `root' superuser during installation or
system configuration, and this password can be changed by the user
himself or by the superuser for any user using the `\cmd{passwd}' command.

\section*{Command Interpreter}

The first characters that you can see at the beginning of a line when you log
in and access the command line interface is the so-called \struct{Shell prompt}.
This prompt asks you to enter the commands. In the simplest case,
in the Bourne shell, it's just a \struct{dollar sign} for a regular user and
a \struct{hash sign} for a superuser. In modern shells, this can be a complex
construct with host and user names, current directory, and so on.
But the meaning of the dollar and hash signs is still the same.

The Shell as a command interpreter that provides a compact and powerful
means of interacting with the kernel and OS utilities. Despite the many
powerful graphical interfaces provided on UNIX-like systems, the command
line is still the most important communication channel for interacting
with them.

All commands typed on a line can be used in command files executed by
the shell, and vice versa. Actions performed in the command interpreter can
then be surrounded by a graphical UI, and the details of their execution,
thus, will be hidden from the end user.

Each time a user logs into the system, he finds himself in the environment of
the so-called home interpreter of the user, which performs configuration
actions for the user session and subsequently carries out interactive
communication with the user. Leaving the user session ends the work of
the interpreter and processes spawned from it. Any user can be assigned any of
the interpreters existing in the system, or an interpreter of his own design.
At the moment, there is a whole set of command interpreters that can be a user
shell and a command files executor:
\begin{itemize}
\item \cmd{sh} is the \struct{Bourne-Shell}, the historical and conceptual
      ancestor of all other shells, developed by Stephen Bourne at AT\&T Bell
      Labs and included as the default shell for Version 7 of Unix.
\item \cmd{csh}~--- \struct{C-Shell}, an interpreter developed at the University
      of Berkeley by Bill Joy for the 3BSD system with a C-like control
      statement syntax. It has advanced interactive tools, task management tools,
      but the command file syntax was different from Bourne-Shell.
\item \cmd{ksh}~--- \struct{Korn-Shell}, an interpreter developed by David Korn
      and comes standard with SYSV. Compatible with Bourne-Shell, includes
      command line editing tools. The toolkit provided by Korn-Shell has been
      fixed as a command language standard in POSIX P1003.2.
\end{itemize}

In addition to the above shells that were standardly supplied with
commercial systems, a number of interpreters were developed,
which are freely distributed in source codes:
\begin{itemize}
\item \cmd{bash}~--- \struct{Bourne-Again-Shell}, quite compatible with
      Bourne-Shell, including both interactive tools offered in C-Shell and
      command line editing.
\item \cmd{tcsh}~--- \struct{Tenex-C-Shell}, a further development of
      the C-Shell with an extended interactive interface and slightly improved
      scripting machinery.
\item \cmd{zsh}~--- \struct{Z-Shell}, includes all the developments of Bourne-Again-Shell
      and Tenex-C-Shell, as well as some of their significant extensions
      (however, not as popular as bash and tcsh).
\item \cmd{pdksh}~--- \struct{Public-Domain-Korn-Shell}, freely redistributable
      analogue of Korn-Shell with some additions.
\end{itemize}

There are many ``small'' shells often used in embedded or mobile systems
such as \struct{ash}, \struct{dash}, \struct{busybox}.

%\section*{Main Concepts}

At the top level, UNIX-like systems can be very convenient for common
users, and they may not even know they are using this type of OS.
For example, currently the most commonly used operating systems are
Linux-based Android systems and UNIX-based Apple systems, in which
the user only sees the user friendly graphical UI.

But beginners who are just starting to learn UNIX-like systems
for administration or development sometimes complain about their complexity.
Don't be afraid~--- actually such systems are based on fairly simple concepts.
There are only three things (three and a half to be exact) you
need to know to be comfortable with any UNIX-like system:
\begin{itemize}
\item[1)] \struct{Users}
\item[2)] \struct{Files}
\item[3)] \struct{Processes}
\item[3.5)] \struct{Terminal lines}
\end{itemize}

The \struct{users} is not very well known to modern users only because we now
have a lot of computer devices with personal access. UNIX was created at a time
when computers were an expensive rarity and a single computer was used
by many users. As a consequence, from the beginning, \struct{UNIX} had
\struct{strong security policies} and \struct{restrictions on permissions}
for users.

And now on UNIX-like systems, we have dozens of users and groups,
even if hidden by an autologin machinery. And most of them are so-called
\struct{pseudo-users}, which are needed to start system services. As we will see
later, they are required by architecture, since it is on the permissions
of users and groups that the system is built to control access to system
resources (processes and files).

If we are talking about ordinary users, they can log in with a \struct{username}
and \struct{password} and interact with the applications installed on the system.
Each user has full permissions only in their home directory and limited
access rights to files and directories outside of it. This can be viewed
as foolproof~--- common users cannot destroy anything on the system just
because they do not have such permissions. Moreover, they \struct{cannot view}
another user's home directory or protected system files and directories.
To perform system administration tasks, the system has a special
\struct{superuser} (generally called ``\struct{root}'') with extra-permissions.

At the system level, each user or group looks like an integer number:
a~user identifier (\struct{UID}) and a group identifier (\struct{GID}).

\struct{Files} are the next important thing for UNIX-like systems. Almost
all system resources look like files, including devices and even
processes on some systems. And the basic concepts have been the same
since the beginning of the UNIX era. We have a hierarchical file system
with a single root directory. All resources, including file systems
existing on devices or external network resources, are attached to this
file system in separate directories~--- this operation is called
``mount''. On the other hand, you can access a device (real or virtual)
as a stream of bytes and work with it like a regular file. All files and
directories are owned by users (real or pseudo) and groups, and read,
write, and execute access to them is controlled by permissions.

A \struct{process} is a program launched from an executable file. Each process
belongs to a user and a group. The relationship between the owners of
processes and resources determines the access rights according to
the resource permissions. All processes live in a hierarchical system based
on parent-child relationships. There is an initial process on the system
called ``\cmd{init}'' that is started at boot. All system services are started
from this initial process.

There are fundamentally two types of processes in Linux~--- foreground and
background:
\begin{itemize}
\item \struct{Foreground processes} (also referred to as interactive processes)~---
      these are initialized and controlled through a terminal session.
      In other words, there has to be a user connected to the system to start
      such processes; they haven’t started automatically as part of the system
      functions/services.
\item \struct{Background processes} (also referred to as non-interactive/automatic
      processes)~--- are processes not connected to a terminal; they don’t
      expect any user input. System services are always background processes.
\end{itemize}

And finally~--- interactive foreground processes must be attached to
the terminal session through the terminal line. At the time of the creation of
UNIX, a TTY (teletype) device (originally developed in the 19th century),
was the primary communication channel between the user and the computer.
It was a very simple interface that worked with a stream of bytes encoded
according to the ASCII character set. The connection is made via a serial
interface (for example RS232) with a fixed set of connection speeds.

%This interface is still the main user interface for UNIX-like systems.
This interface is still the main user interface for UNIX-like systems.
The implementation of each new form of user interaction, such as
full-screen terminals, graphics systems, and network connections, all
started with the implementation of a simple TTY command line interface.
Moreover, as we will see, this interface abstraction gives us a very
powerful and flexible mechanism for communication between programs,
possibly without human intervention.

\end{document}

000_intro.tex                  010_input_output_redirect.tex
001_history.tex                011_shell_settings.tex
002_open_free.tex              012_keystrokes.tex
003_main_concepts.tex          013_utilities.tex
004_components.tex
005_your_system.md
006_command_inerpreter.md