\documentclass[12pt]{report}
\usepackage{cmap}

\usepackage[T1]{fontenc}
\usepackage[utf8]{inputenc}

\usepackage[english]{babel}
%\usepackage[english,russian]{babel}

\usepackage{hyperref}

\usepackage{amsthm}
\usepackage{amsmath}
\usepackage{amssymb}
\usepackage{amsfonts}
\usepackage{amsthm}
\usepackage{enumerate}
%\usepackage{enumitem}
\usepackage{tikz}
\usepackage{xcolor}
\usepackage{textcomp}

\usepackage{listings}

\usepackage{comment}

\newcommand{\struct}[1]{\textcolor{blue}{#1}}
\newcommand{\Rim}[1]{\uppercase\expandafter{\romannumeral#1}}
\newcommand{\cmd}[1]{\textcolor{blue}{\tt #1}}

\definecolor{mverb}{HTML}{C0E8F8}

\lstnewenvironment{code}[1]{%
  \lstset{backgroundcolor=\color{#1},
  tabsize=10,
  frame=single,
  showlines=true,
  framerule=0pt,
  basicstyle=\ttfamily,
  columns=fullflexible}}{}

\parskip5pt
\parindent0pt

\title% (optional, use only with long paper titles)
{UNIX AND LINIX\\IN INFOCOMMUNICATION\\Week 4}

\author % (optional, use only with lots of authors)
{O.~Sadov%,\\ \texttt{tit@astro.spbu.ru}
}
% - Give the names in the same order as the appear in the paper.
% - Use the \inst{?} command only if the authors have different
%   affiliation.

%%%\institute{ITMO}
%  Университет Информационных Технологий, Механики и Оптики\\
%  кафедра Телекоммуникационных Систем

% - Use the \inst command only if there are several affiliations.
% - Keep it simple, no one is interested in your street address.

\date% (optional, should be abbreviation of conference name)
{Aug 23, 2020 10:33 PM}


\begin{document}
\maketitle
\input{../017_files}
\section*{File Commands}

UNIX tools support a standard set of commands for working with files and
directories:
\begin{itemize}
\item \cmd{ls}~--- list directory contents. Let's look in `\cmd{man ls}'.
      We can simply specify files and directories as arguments and view
      the listing in different ways according to the options.

      Ok, let's take a look at our current directory~--- it's just
      `\cmd{ls}' without arguments. As we remember, after logging in,
      this is the home directory.
\begin{code}{mverb}
ls
\end{code}
We see some directories, but we don't see, for example, shell startup
files. No problem, let's run:
\begin{code}{mverb}
ls -a
\end{code}
We can see the shell startup files and more~--- the directories ``\cmd{.}'' (``point'')
(current) and ``\cmd{..}'' (``double point'') (top level) are also visible.
Because that means ``all'' files and directories, including hidden ones.
Hidden files in UNIX are just a naming convention~--- names must begin with
a period. It is not an attribute as it is on Microsoft systems. Initially
it was just a trick in the `\cmd{ls}' utility to hide the current and top
directories, and then it came to be used as a naming convention to hide
any file or directory.

Also we can see directory listing recursively:
\begin{code}{mverb}
ls -R
\end{code}
Another very important option is the ``long list'' (``\cmd{-l}''):
\begin{code}{mverb}
ls -l
\end{code}
We see a table with information about the file/directory in the corresponding
lines.
\begin{itemize}
\item The first column is the file attribute. The first letter is the file
      type: ``\cmd{-}'' (``dash'') is a regular file, ``\cmd{d}'' is a directory,
      and so on. Then we can see read, write, and execute permissions for three
      user groups: owner, owner group, and everyone else. Once again, we see
      the difference between UNIX and Microsoft. In the first case it is
      an attribute, in the second case executability is just a naming
      convention: '.com', '.exe', '.bat'.
\item Some mystery column that we will discuss later.
\item Then we can see owner and owner group, size of file, time of
      modification and the name of file. 
\end{itemize}
\item \cmd{pwd}~--- print name of current/working directory
\item \cmd{cd}~--- change directory. Without arguments -- to home firectory.
\item \cmd{cp}~--- copy files and directories. Most interesting option
                   is `\cmd{-a|-\mbox{}-archive}' with create recursive
                   archive copy with preserving of permissions, timestamps, etc\ldots
\item \cmd{mv}~--- move (rename) files and directories.
\item \cmd{rm}~--- remove files or directories.
\begin{code}{mverb}
rm -rf ...
\end{code}
means recursive delete without asking for confirmation.
\item \cmd{mkdir}~--- make directories. If any parent directory does not exist,
      you will receive an error message:
\begin{code}{mverb}
mkdir a/b/c
mkdir: cannot create directory 'a/b/c': No such file or directory
\end{code}
To avoid this, use the \cmd{-p} option:
\begin{code}{mverb}
mkdir -p a/b/c
\end{code}
\item \cmd{rmdir}~--- remove empty directories. If directory is not empty,
      you will receive an error message. Nowadays, running `rm -rf something...'
      is sufficient in this case. But in the old days, when `\cmd{rm}' did not have
      a recursive option, to clean up non-empty directories, you had to create
      a shell script with `rm's in each subdirectory and the corresponding
      `rmdir's.
\item \cmd{ln}~--- make links between files. Links are a very specific file
      type in UNIX and we will discuss them in more detail. If we look at
      the man page for the `ln' command, we see a command very similar to `cp'.
      But let's take a closer look:
\begin{code}{mverb}
cat > a
ln -s a b
ln a c
cat b; cat c
\end{code}
At the moment everything looks like a regular copy of the file, but let's
try to change something in the one of them:
\begin{code}{mverb}
cat >> c
cat a; cat b
\end{code}
Wow, all the other linked files have changed too! We are just looking at
the same file from different points, and changing one of them will change
all the others. And in this they all seem to be alike. But let's try
to delete the original file:
\begin{code}{mverb}
rm a
cat b; cat c
\end{code}
In the first case, we can still see the contents of the original file,
but in the second case, we see an error message. Simply because the first
is a so-called hard link and the second is symbolic. We can see the
difference between the two in the long ls list:
\begin{code}{mverb}
ls ?
\end{code}
And we can restore access to the content for the symbolic link by simply
recreating the original file:
\begin{code}{mverb}
ln b a
cat c
\end{code}
Another difference between them is the impossibility of creating a hard link
between different file systems and the possibility of such a linking
for soft links. For more details on internal linking details, see
the corresponding
lecture.
\href{under_the_hood/links.md}{``Under the Hood''}%[``Under the Hood''](under_the_hood/links.md)
\end{itemize}

\subsection*{Permissions}

And finally, let's discuss file permissions. As we remember, we have
the owner user, the owner group and all the others, as well as read, write
and execute permissions for such user classes. And we have the appropriate
command to change these permissions:
\begin{code}{mverb}
chmod [-R] [ugoa][-+=](rwx)
\end{code}
And as we understand it, permissions are just a bit field. As far as
we understand, permissions are just a bitfield and in some cases it might be
more useful to set them in octal mode~--- see for information on this.

\href{under_the_hood/octal_mode.md}{``Under the Hood''}%[``Under the Hood''](under_the_hood/octal_mode.md)

You can also change the owner and group for a file or directory
by command `\cmd{chown}'.
\begin{code}{mverb}
man chown - change file owner and group
\end{code}
But keep in mind~--- for security reasons, only a privileged user
(superuser root) can change the owner of a file. The common owner of a file
can change the group of a file to any group that owner is a member of:
\begin{code}{mverb}
chown :group file...
chgrp group file...
\end{code}

\input{../019_text_viewers}
\section*{Text Editors}

OK. We can create a file using the `\cmd{cat}' utility and view the file using
a viewer. But what if we need to change something, especially if we only have
a TTY interface? And it is possible~--- we have a so-called line editor
named `\cmd{ed}'. The only interface for such an editor is the command line:
%[Pr-n 6 slide 5]
(\url{http://sdn.ifmo.ru/education/courses/free-libre-and-open-source-software/lectures/lecture-6}).

So let's try to edit new file.
\begin{code}{mverb}
$ ed tst
tst: No such file or directory
\end{code} % $
At first~--- we will add some lines:
\begin{code}{mverb}
i
1 2
3 4
\end{code}
and we must end our input with one `dot' per line.
\begin{code}{mverb}
.
\end{code}
Let's take a look at our file, moving to the first line:
\begin{code}{mverb}
1
1 2

3 4

?
\end{code}
Seems good. Now we can add something to the end:
\begin{code}{mverb}
a
5 6
.
1
1 2

3 4

5 6

?
\end{code}
OK~--- we have 3 lines in the file now.
And finally~--- let's try to make a magic pass:
\begin{code}{mverb}
1,$s/\(.\) \(.\)/\2 \1/
1
2 1

4 3

6 5

?
\end{code} % $
This means: from the first to the last line, replace the lines where we have
two separate letters separated by a space, exchanging those letters with places.
And now `write' and `quit':
\begin{code}{mverb}
w
12
q
\end{code}
Let's check the result:
\begin{code}{mverb}
$ cat tst
2 1
4 3
6 5
\end{code} % $

But for what purposes can you use a line editor now that we have full-screen
editors with a user-friendly interface? Of course, you can imagine a situation
where your full-screen environment is broken and only the line editor will be
the salvation. And in general I had such situations. But the main use case
for \cmd{ed} is for automatic editing in scripts. Anything you need to change
in the text data can be done with this editor, including sophisticated regex
search and replace.

Moreover, we have a `\cmd{sed}'~--- stream editor, for editing text data
in pipelines:
\begin{code}{mverb}
$ sed 's/\(.\) \(.\)/\2 \1/' < tst
1 2
3 4
5 6
\end{code} % $

As you can see, the original file does not change, all changes are simply
sent to standard output:
\begin{code}{mverb}
$ cat tst
2 1
4 3
6 5
\end{code} % $

But UNIX-like systems also have full-screen editors, which can also be
confusing for beginners. It was developed by Berkeley student Bill Joy
for BSD initially as a visual mode for a line editor. It is a very fast
and lightweight editor that is part of the  Single Unix Specification and
the POSIX, which found on every UNIX-like system. The \cmd{VI} editor works
on all types of terminals and generally requires only a conventional letter
keyboard. You can work with it without the arrow keys, PgUp/Down or
anything similar. There are actually very small keyboards out there that
are optimized for `\cmd{vi}'.

But to work on it, you need to understand the basic concept of this editor:
it can be in three states
%[Pr-n 6 slide 6]
(\url{http://sdn.ifmo.ru/education/courses/free-libre-and-open-source-software/lectures/lecture-6}).

Immediately after launch, we find ourselves in the usual \struct{command mode}
and can switch to \struct{editing mode}, for example, by pressing
the ``{\tt [Insert]}'' key. As we can see, the mode status in the lower left
corner has changed to `-\mbox{}- INSERT -\mbox{}-', and now we can edit our file.
Pressing {\tt Insert} again will change the state mode to
`-\mbox{}- REPLACE -\mbox{}-' and vice versa. Exit the editing mode by pressing
{\tt ESC}. The third mode can be accessed by pressing the colon key in command
mode. This is \struct{`ed' mode}. In this mode, we can use the normal
`\cmd{ed}' line editor commands and finish them with {\tt ENTER}.

In command mode, you can find something with regex by slash and question marks,
as in the `\cmd{less}' viewer. In improved VI (\cmd{vim}), you can use very
useful visual mode by pressing {\tt V}. After that you can delete the selection
with `\cmd{d}' or just copy it with `{\cmd{y}' (yank). Then you can paste it
anywhere with `\cmd{p}' (paste). Moreover, you can use {\tt [Ctrl-V]} to select
a vertical box. To exit visual mode, simply press {\tt ESC}.

Also you may look to \cmd{vimtutor}~--- a guide to Vim can be useful
for beginners.

And the second most common editor is \cmd{Emacs}. This Richard Stallman's editor
was the starting point for the GNU Project, along with GCC and UNIX utilities.
EMACS means, for example, ``Editing MACroS''. An apocryphal hacker koan alleges
that the program was named after Emack\&Bolio's, a popular Cambridge ice cream
store. But overall it is a really very powerful editor with macro extensions,
allowing the user to override any keystrokes to launch the editor program.
But unlike other editors that support macro-extensions, in Emacs they are
implemented using the LISP programming language embedded on editor.
At the time, LISP was very popular in artificial intelligence in
the United States, and Stallman, who worked at the MIT Artificial Intelligence
Lab, chose it as the editor extension language.

This implementation allows many LISP-based applications to be developed,
including a user-friendly interface for developers in various programming
languages. Usually Emacs is a text editor with a simple graphical interface.
But it can only be run in a text environment. The most commonly used
keystrokes are:\\
\verb|C-c C-x|~--- exit\\
\verb|C-h t|~--- tutorial\\
\verb|C-h i|~--- info

\medskip
If you feel overwhelmed by the difficulty of Emaccs, you can see a personal
psychoanalyst: \cmd{M-x doctor}. It would spoil the fun and hurt your recovery
to say too much here about how the doctor works. But when you're ready,
you may try to find the well-known Turing-test related AI program ELIZA
on WikiPedia.

Also in the UNIX/Linux world, there are many other editors that may be more
convenient for you, such as:
\begin{itemize}
\item \cmd{joe}, \cmd{nano}~--- simple text editors or
\item \cmd{gedit}, \cmd{kate} -- text editors from Gnome and KDE projects
\end{itemize}

\section*{Advanced Text Utilities}

\subsection*{Searching} % ## Searching

If we are talking about text data, finding some text is a common task.
And in fact, these are two separate tasks~--- to find some text inside a file
or text stream and to find a file, for example, by name in some directories.

For the first task, we have the `\cmd{grep}' utility which print lines matching
a pattern.
\begin{code}{mverb}
man grep
\end{code}
Both fixed strings and regular expressions can be used as a pattern.
Also you can do recursive search.

Another commonly used command is `\cmd{find}'~--- search for files
in a directory hierarchy.
\begin{code}{mverb}
man find
\end{code}

You must set the starting point~--- the directory to start the search or
starting points if you are interested in several directories and expressions
with search criteria and actions. You may search by name with using of standard
shell matching patterns, by time of modification or access, by size, by user
and group, by permissins, file type, etc. You can use logical operators such
as ``\struct{and}'', ``\struct{or}'' and ``\struct{not}'' in your expressions.

Also you can do some actions when you find something that matches the criteria.
The default action is `\cmd{print}'. You can also use formatted printing,
list of found files, delete them, and execute commands with them.
In `\cmd{exec}' actions, you can use curly braces to insert the name of
the found file. But keep in mind~--- you must end your command with a semicolon,
and to avoid interpreting this Shell character, you must escape it
with a `slash' (`/').

But the main drawback of `\cmd{find}' is the long execution time if you are
looking in large deep directories. And to speed up this process, you can use
the `\cmd{locate}' utility. It finds files by name from databases prepared
by `\cmd{updatedb}' and does it incredibly fast. But you have to understand~---
`\cmd{updatedb}' is started automatically by the cron service at night.
And if you only install the `\cmd{lookup}' toolkit or want to find something
in the changed filesystem at this time~--- you have to update this database
manually by running `\cmd{updatedb}'.

\subsection*{Utilities for Manipulation with a Text Data}
%## Utilities for manipulation with a text data

Another operation that we often need is comparing files or directories.
And we have some tools for this.
\begin{code}{mverb}
man cmp
\end{code}
The `\cmd{cmp}' utility compares the two files byte-by-byte and reports
the position from which we have a difference. By this way we can compare
binaries.

To compare text files `\cmd{diff}' utility can be used:
\begin{code}{mverb}
man diff
\end{code}
We can compare files, directories with the `\cmd{-\mbox{}-recursive}' option.
We can get the output as a set of commands for the `\cmd{ed}' editor or
the `\cmd{patch}' utility. This method of propagating changes was the first
in the development of projects in the UNIX ecosystem and is still useful today.

Another important action with text data is sorting, and we have
the `\cmd{sort}' utility which sort lines of text files:
\begin{code}{mverb}
man sort
\end{code}
To eliminate duplicate lines, we have the uniq utility, but first we have
to sort our text stream:
\begin{code}{mverb}
sort file | uniq
\end{code}

We may output the first/last part of files by `\cmd{head}' and `\cmd{tail}'
utilities. By default is the first or last 10 lines of standard input, or
each FILE from arguments to standard output. You can set another number
of lines as an option:
\begin{code}{mverb}
tail -15
\end{code}
Also in `\cmd{tail}' you can use the `\cmd{-\mbox{}-follow}' option to display
the appended data as the file grows.

More that, from text lines you can cut some fields, separated by some
kind of separators by `\cmd{cut}' utility.

Also you can join lines of two files on a common field by `\cmd{join}' utility
and merge lines of files by `\cmd{paste}'.

And finally, we have `\cmd{awk}', a scanning and templating language that can
do this and other complex work on text files or streams.

\section*{Basic Network Utilities} % ## Basic network utilities

From the very beginning of the development of computer communication
technologies, UNIX has been closely associated with them. Historically,
the first worldwide network to operate over conventional dial-up telephone
lines was created in the late 1970s at At\&T and called \cmd{UUCP}~---
``UNIX to UNIX copy''. And in 1979, two students at Duke University,
\struct{Tom Truscott} and \struct{Jim Ellis}, originated the idea of
using Bourne shell scripts to transfer news and mail messages on a serial line
UUCP connection with nearby University of North Carolina at Chapel Hill.
Following public release of the software in 1980, the mesh of UUCP hosts
forwarding on the Usenet news rapidly expanded and named UUCPnet.

Technically, in the beginning, these could be dial-up modems, simply attached
to the telephone tubes with suction cups which makes connects on hundreds of
bits per second speed with very unstable connection. Even so, on this stage
UNIX offered a fully functional network with the ability to remotely execute
commands and transfer data over a complex mesh network topology.

UUCP provided just two main utilities:
\begin{itemize}
\item \cmd{uucp}~--- system-to-system copy
\item \cmd{uux}~--- remote command execution
\end{itemize}

It was a very simple addressing scheme with no dynamic routing or anything
similar, and the command to do something on a remote machine with files hosted
on other machines might look like this:
\begin{code}{mverb}
uux 'diff sys1!~user1/file1 sys2!~user2/file2 >!file.diff'
\end{code}
Fetch the two named files from system sys1 and system sys2 and execute
\cmd{diff} putting the result in file.diff in the current directory.
It's funny, this addressing is still supported, for example,
by the `\cmd{sendmail}' mail system, which adds some complexity.

\subsection*{Traditional Network Utilities} % ## Traditional network utilities

In the world of TCP/IP Network, other programs have been developed that are
still relevant in some cases, classical Internet programs:
\begin{itemize}
\item \cmd{telnet}~--- user interface to the TELNET protocol
\item \cmd{ftp}~--- ARPANET file transfer program
\item \cmd{mail}~--- send and receive mail
\end{itemize}
Again we have a tool for remote execution and a tool for data transfer.

Generally, telnet just gives us a connection to the TELNET protocol server:
\begin{code}{mverb}
man telnet
\end{code}
It's just a CLI for another host and this protocol still used for access
to some hardware devices. Moreover, you can use it for debugging by connecting
to other servers by choosing of TCP server's port. For example HTTP:
\begin{code}{mverb}
$ telnet google.com 80
Trying 173.194.73.101...
Connected to google.com.
Escape character is '^]'.
GET /index.html HTTP/1.1
\end{code} % $
To switch to telnet command mode, press the ``\struct{Ctrl-]}'' key.
Here we can ask for help and exit, for example, if the program on the other
side is frozen:
\begin{code}{mverb}
telnet> h
Commands may be abbreviated.  Commands are:
close			close current connection
logout		forcibly logout remote user and close the connection
display	display operating parameters
mode				try to enter line or character mode ('mode ?' for more)
open				connect to a site
quit				exit telnet
send				transmit special characters ('send ?' for more)
set					set operating parameters ('set ?' for more)
unset			unset operating parameters ('unset ?' for more)
status		print status information
toggle		toggle operating parameters ('toggle ?' for more)
slc					set treatment of special characters

z							suspend telnet
environ	change environment variables ('environ ?' for more)
telnet> q
Connection closed.
\end{code}

FTP or File Transfer Protocol is another well-known part of the networked
world of the Internet. It is still supported by some internet servers and
is also built into some devices. We can access the FTP server through a regular
web browser as well as through the \cmd{ftp} utility:
\begin{code}{mverb}
man ftp
\end{code}

In some cases, the latter variant is preferable, because, for example,
we may want to restore a file or upload/download many files. First, we have
to log into the FTP server. Let's try to do this as an anonymous user:
\begin{code}{mverb}
$ ftp ftp.funet.fi
Name (ftp.funet.fi:user): ftp
331 Any password will work
Password:
\end{code} % $

In this case any password will work, but often FTP-server wait email
address as a password.

FTP has its own command line interface where we can ask for help:
\begin{code}{mverb}
ftp> ?
Commands may be abbreviated.  Commands are:

!       dir         mdelete  qc        site
$       disconnect  mdir     sendport  size
account exit        mget     put       status
append  form        mkdir    pwd       struct
ascii   get         mls      quit      system
bell    glob        mode     quote     sunique
binary  hash        modtime  recv      tenex
bye     help        mput     reget     tick
case    idle        newer    rstatus   trace
cd      image       nmap     rhelp     type
cdup    ipany       nlist    rename    user
chmod   ipv4        ntrans   reset     umask
close   ipv6        open     restart   verbose
cr      lcd         prompt   rmdir     ?
delete  ls          passive  runique
debug   macdef      proxy    send
ftp>
\end{code} % $

We can first determine our current directory, and as we understand it,
we have two current directories: remote and local. We can get the remote
directory with the standard `\cmd{pwd}' command. To get the current local
directory we can use the same command preceded by an exclamation mark.
This means~--- call this command on the local computer. You may change
directory remotely by `\cmd{cd}' and local directory by `\cmd{lcd}'.

We can get a list of remote directoriy using the well-known `\cmd{ls}' command.
And what about local `\cmd{ls}'? Yes~--- just preced it by an exclamation mark.
If you have sufficient permissions, you can download file by `\cmd{get}' command
and upload by `\cmd{put}', but only a single file. If you want to work
with multiple files, you will need to use the `\cmd{mget}'/`\cmd{mput}' commands.

In this case, it makes sense to disable the questions about confirming
operations using the prompt command. Also switching to binary mode using
the bin command can be important for the Microsoft client system.
Otherwise, you may receive a corrupted file.

Finally, you can use the `\cmd{reget}' command to try to continue downloading
the file after an interrupted transfer. And the `\cmd{hash}' command toggle
the `hash' printing for each transmitted data block, which can be informative
if the connection to the server is poor.

Another useful scripting program is `\cmd{mail}', which is a simple command
line client for sending email:
\begin{code}{mverb}
$ mail user@localhost
Subject: test
This is a test!
.
\end{code} % $
The mail message must end with one `dot' per line.

\input{../023_internet_tools}
\end{document}
